\documentclass[paper=a4, fontsize=11pt]{scrartcl}

\usepackage[T1]{fontenc} 
\usepackage[english]{babel} 
\usepackage{amsmath}
\usepackage{amsfonts}
\usepackage{amsthm}
\usepackage{amssymb}
\usepackage{changepage}
\usepackage{titlesec}
\usepackage{sectsty} 
\sectionfont{\centering \normalfont \scshape}
\subsectionfont{\normalfont}
\subsubsectionfont{\normalfont}

\usepackage{fancyhdr} 
\pagestyle{fancyplain} 
\fancyhead{} 
\fancyfoot[L]{} 
\fancyfoot[R]{} 
\fancyfoot[C]{\thepage} 
\renewcommand{\headrulewidth}{0pt} 
\renewcommand{\footrulewidth}{0pt} 
\setlength{\headheight}{13.6pt} 

\usepackage{mathtools}
\usepackage{enumitem}
\newcommand{\subscript}[2]{$#1 _ #2$}

\newcommand{\horrule}[1]{\rule{\linewidth}{#1}} 
\newcommand{\ball}[2]{$B({#1};{#2})$}
\newcommand{\overbar}[1]{
	\mkern 1.5mu \overline{\mkern-1.5mu\raisebox{0pt}[\dimexpr\height+0.5mm\relax]{$#1$}\mkern-1.5mu}\mkern 1.5mu
}

\title{	
	\normalfont \normalsize 
	\textsc{Konkuk University Dept. Of Physics} \\ [25pt] %Konkuk University Dept. of Physics
	\horrule{1pt} \\[0.4cm] 
	\huge Algebraic Topology, Exercises 1 \\
	\vspace{0.1in}
	\Large 2019 Spring Semester
	\horrule{1pt} \\[0.4cm] 
}

\author{Youngwan Kim} 
\date{\normalsize\today} 
%\newtheorem{theorem}{Thm}
%\newtheorem{definition}{Def}
%\newtheorem{examples}{Examples}
%\newtheorem{example}{Ex}
%\newtheorem{lemma}{Lem}
%\newtheorem{corollary}{Cor}
\newtheorem*{remark}{Remark}
\newtheorem*{recall}{Recall}
\newtheorem{problem}{Problem}

\begin{document}
	
\maketitle	

\begin{recall}
	Recall that a map $f:X \to Y $ is \textbf{homotopic} to a map $g:X\to Y$ if there is a map $F:X\times[0,1]\to Y$ such that $F(x,0)=f$ and $F(x,1)=g$ for $\forall x\in X$. (no relative conditions)\\
\end{recall}

\begin{problem}
	Show that if the topological spaces $X$ and $Y$ are homeomorphic and $X$ is simply connected, then so is $Y$.\\
\end{problem}


\begin{proof}
	If $X \cong Y $ then $\pi_1(X) \simeq \pi_1(Y)$. As $\pi_1(X) = 0$ and $\pi_1(X) \simeq \pi_1(Y)$\, it implies that $\pi_1(Y) = 0$.\\
\end{proof}

\begin{problem}
	Let $n$ be a positive integer. Let $f:X\to S^n$ and $g:X\to S^n$ be maps. Suppose that $f(x)\neq -g(x)$ for any $x \in X$. Show that $f$ is homotopic to $g$.\\
\end{problem}

\begin{proof}
	Define a map $ F:X \times [0,1] \to S^n$ such that \\
	\begin{equation}\nonumber
		F(x,t) = \frac{(1-t)\cdot f(x)+t \cdot g(x)}{|(1-t)\cdot f(x)+t \cdot g(x)|}
	\end{equation}
	then we can check that\\
	\begin{equation}\nonumber
		\begin{cases}
		F(x,0)=\frac{f(x)}{|f(x)|}=f(x) & \\
		F(x,1)=\frac{g(x)}{|g(x)|}=g(x)
		\end{cases}
	\end{equation}\\
	as $f(x),g(x)\in S^n$. Also as $f(x)\neq -g(x)$ for all $x \in X$, such $F(x,t)$ is well defined as $|(1-t)\cdot f(x)+t \cdot g(x)| \neq 0$ for all $(x,t) \in X \times [0,1]$.\\
\end{proof}

\begin{remark}
	The trick of normalizing line segments are well used as a method of constructing a homotopy between maps which maps to $S^n$, due to its property.\\
\end{remark}

\begin{problem}
	Let $X=\{x\in \mathbb{R}^n:1\leq |x| \leq 2\}$. Let $f:X\to X$ be a map defined by $f(x)=x/|x|$. Show that $f$ is homotopic to the identity map $id:X\to X$.\\
\end{problem}

\begin{proof}
	Let $F:X\times [0,1] \to X$ as 
	\begin{equation}\nonumber
		F(x,t) = (1-t) \cdot \frac{x}{|x|} + t \cdot x
	\end{equation}
	Then $F(x,0)=f$ and $F(x,1)=id$, thus as there exists such homotopy $F$, $f \simeq id$.\\
\end{proof}

\begin{remark}
	For the above problem, $F(x,t)=\frac{x}{|x|^t}$ is also a homotopy.\\
\end{remark}

\begin{problem}
	Show that $\pi_1(X\times Y,(x_0,y_0))$ is isomorphic to the direct product $\pi_1(X,x_0)\times\pi_1(Y,y_0)$.\\
\end{problem}

\begin{proof}
	Let $p_X:X\times Y \to X$ and $p_Y:X \times Y \to Y$ be projections, i.e, for $\forall (x,y) \in X\times Y$, the maps are defined as $p_X(x,y)=x$ and $p_Y(x,y)=y$. Now let $\phi : \pi_1(X\times Y,(x_0,y_0)) \to \pi_1(X,x_0)\times\pi_1(Y,y_0)$ for all $[\alpha] \in \pi_1(X\times Y,(x_0,y_0))$ which maps as,
	\begin{equation}\nonumber
		\phi([\alpha]) = ([p_X \circ \alpha],[p_Y \circ \alpha])
	\end{equation}
	$\phi$ is well defined.
	Such $\phi$ is obviously a bijection, as we can define $\phi^{-1}:\pi_1(X,x_0)\times\pi_1(Y,y_0) \to \pi_1(X\times Y,(x_0,y_0))$. We just have to show that it is a homomorphism. For any $[\beta],[\gamma]\in\pi_1(X\times Y,(x_0,y_0))$  \\
	\begin{equation}\nonumber
		\begin{split}
			\phi([\alpha][\beta]) = \phi([\alpha\beta]) &= ([p_X \circ \alpha \beta] ,[p_Y \circ \alpha \beta]) \\
			&= ([p_X \circ \alpha][p_X \circ \beta],[p_Y \circ \alpha][p_Y \circ \beta]) \\
			&= ([p_X \circ \alpha],[p_Y \circ \alpha])([p_X \circ \beta],[p_Y \circ \beta]) \\
			&= \phi([\alpha]) \phi([\beta])			
		\end{split}
	\end{equation}\\
\end{proof}

\begin{problem}
	Prove that the product of simply connected spaces is simply connected.\\
\end{problem}



\begin{proof}
	Using the results of \textbf{Problem 4}, if $\pi_1(X) = 0$ and $\pi_1(Y) = 0$ then $\pi_1(X \times Y) \cong \pi_1(X) \times \pi_1(Y) = 0$. Also as both $X,Y$ are path connected, $X\times Y$ is also path connected, as there exists any path $\gamma(s)=(\alpha(s),\beta(s))$ for all $(x_0,y_0),(x_1,y_1)\in X$ such that $\gamma(0)=(x_0,y_0)$ and $\gamma(1)=(x_1,y_1)$, where the existence of $\alpha$ and $\beta$ is guranteed as $X,Y$ are path connected. Thus as $\pi_1(X\times Y) = 0$ and $X \times Y$ is path connected, it is simply connected.\\
\end{proof}

\begin{problem}
	Prove that if $n \geq 3$, then $\mathbb{R}^n \setminus \{0\}$ is simply connected. (Hint : use the fact that $S^{n-1}$ is simply connected.)\\
\end{problem}

\begin{proof}
%	Consider a map $p : \mathbb{R}^n \setminus \{0\} \to S^{n-1}  $ such that \\
	For any $[\alpha] \in \mathbb{R}^n \setminus \{0\}$, by some homotopy, let it $F$, $\alpha \simeq \frac{\alpha}{|\alpha|}\in S^{n-1}$. In $S^{n-1}$, $ \alpha/|\alpha| \simeq 1$ by a homotopy, let it $G$ as $S^{n-1}$ is simply connected for $n\geq 3$. As $S^{n-1} \subset \mathbb{R}^n$, we can say that $\alpha \simeq \alpha/|\alpha| \simeq 1$, thus $\mathbb{R}^n \setminus \{0\}$ is also simply connected. \\
\end{proof}

\begin{remark}
	This problem can be easily solved using retract deformations, or considering homotopy equivalence of $S^n-1$ and $\mathbb{R}^n \setminus \{0\}$. \\
\end{remark}

\begin{problem}
	Let $A$ be a subspace of $X$ and $j:A\xhookrightarrow{}X$ be the inclusion map. Let a map $r:X\to A$ be a retraction of $X$ onto  $A$, that is $r \circ j = id_A$. Prove the following,\\
	\begin{enumerate}[label=(\alph*)]
		\item $j_*:\pi_1(A,b)\to \pi_1(X,b)$ is one-to-one.		
		\item $r_*:\pi_1(X,b)\to \pi_1(A,b)$ is onto.
		\item If $X$ is simply connected, then so is $A$.\\
	\end{enumerate}
\end{problem}

\begin{proof}
	As $r \circ j = id_A$, $(r \circ j)_* = (id_A)_* \implies r_* \circ j_* = id_{\pi_1(A)}$. This implies (a) and (b). Also assuming $X$ is simply connected, i.e, $\pi_1(X,b) = 0$ then due to (a) and (b), so does $\pi_1(A) = 0$. Path connectedness of $A$ is also guranteed as $r:X\to A$ is a continuous map and $X$ is simply connected.\\
\end{proof}

\end{document}