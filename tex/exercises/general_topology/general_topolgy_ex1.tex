\documentclass[paper=a4, fontsize=11pt]{scrartcl}

\usepackage[T1]{fontenc} 
\usepackage[english]{babel} 
\usepackage{amsmath}
\usepackage{amsfonts}
\usepackage{amsthm}
\usepackage{amssymb}
\usepackage{changepage}
\usepackage{titlesec}
\usepackage{sectsty} 
\sectionfont{\centering \normalfont \scshape}
\subsectionfont{\normalfont}
\subsubsectionfont{\normalfont}

\usepackage{fancyhdr} 
\pagestyle{fancyplain} 
\fancyhead{} 
\fancyfoot[L]{} 
\fancyfoot[R]{} 
\fancyfoot[C]{\thepage} 
\renewcommand{\headrulewidth}{0pt} 
\renewcommand{\footrulewidth}{0pt} 
\setlength{\headheight}{13.6pt} 

\usepackage{mathtools}
\usepackage{enumitem}
\newcommand{\subscript}[2]{$#1 _ #2$}

\newcommand{\horrule}[1]{\rule{\linewidth}{#1}} 
\newcommand{\ball}[2]{$B({#1};{#2})$}
\newcommand{\overbar}[1]{
	\mkern 1.5mu \overline{\mkern-1.5mu\raisebox{0pt}[\dimexpr\height+0.5mm\relax]{$#1$}\mkern-1.5mu}\mkern 1.5mu
}

\title{	
	\normalfont \normalsize 
	\textsc{Konkuk University Dept. Of Physics} \\ [25pt] %Konkuk University Dept. of Physics
	\horrule{1pt} \\[0.4cm] 
	\huge General Topology, Exercises 1 \\
	\vspace{0.1in}
	\Large 2019 Spring Semester
	\horrule{1pt} \\[0.4cm] 
}

\author{Youngwan Kim} 
\date{\normalsize\today} 
%\newtheorem{theorem}{Thm}
%\newtheorem{definition}{Def}
%\newtheorem{examples}{Examples}
%\newtheorem{example}{Ex}
%\newtheorem{lemma}{Lem}
%\newtheorem{corollary}{Cor}
\newtheorem*{remark}{Remark}
\newtheorem*{recall}{Recall}
\newtheorem{problem}{Problem}

\begin{document}
	
\maketitle	

\begin{problem}
	Show that $[0,1)$ is neither close dnor open in $\mathbb{R}$.\\
\end{problem}

\begin{proof}
	As $int([0,1))=(0,1)$ and $\overbar{[0,1)} = [0,1]$ thus it is neither close or open in $\mathbb{R}$.\\
\end{proof}

\begin{problem}
	Let $A=\{0\}\cup\{\frac{1}{n} | n \in \mathbb{N}\}$. Consider $A$ as a subspace of $\mathbb{R}$. Is it closed or open?\\
\end{problem}

\begin{proof}
	Consider $\mathbb{R}\setminus A$, which becomes
	\begin{equation}\nonumber
		\mathbb{R}\setminus A = (-\infty,0) \cup \left( \bigcup\limits_{n \in \mathbb{N}} \left(\frac{1}{n+1} , \frac{1}{n}\right) \right) \cup (1.\infty)
	\end{equation}
	which is an union of open sets, implying that $\mathbb{R}\setminus A$ is also open. As $\mathbb{R}\setminus A$ is open, $A$ is closed.\\
\end{proof}

\begin{problem}
	Show that for $p \in X$, the singleton $\{p\}$ is closed in $X$. Also show that a finite subset of $X$ is closed in $X$. \\
\end{problem}


\begin{proof}
$ $ \newline
\vspace{-0.15in}
\begin{enumerate}
	\item The singleton is closed in $X$.
	\vskip 0.5ex
	Consider $q \in X \setminus \{p\}$, then there exists $B(q,r)$ where $r=d(p,q)>0$. Then $p \notin B(q,r)$ which implies that $B(q,r)\subset X\setminus \{p\}$. Thus $X \setminus \{p\}$ is open, and $\{p\}$ is thus closed.
	\item A finite subset of $X$ is closed in $X$.
	\vskip 0.5ex
	For any finite subset, let it $P$, it could be expressed as $P=\bigcup\limits_{p \in  P} \{p\}$, which is a finite union of closed sets. As any finite subset is such a finite union of closed sets, it is closed.
\end{enumerate}
\end{proof}

\begin{problem}
	Let us consider $A=\{0,1,2,3\}$ and $B=\{0,1\}$ both as subspaces of $\mathbb{R}$.
	\begin{enumerate}[label=(\alph*)]
		\item Prove that $A$ is closed in $\mathbb{R}$.
		\item Is $B$ open in $\mathbb{R}$?
		\item Is $B$ open in $A$?
		\item Find all open subsets of $A$.\\
	\end{enumerate}
\end{problem}

\begin{proof}
	content...
\end{proof}

\begin{problem}
	A set of the form $\{y \in X | d(x,y) \leq r \}$ is called a closed ball. Show that a closed ball is a closed set. Is the closed ball $\{ y \in X | d(x,y) \leq r \}$ always the closure of the open ball $B(x,r)$?\\
\end{problem}

\begin{proof}
	$ $ \newline
	\vspace{-0.15in}
	\begin{enumerate}
		\item Closed ball is closed. 
		\vskip 0.5ex
		
		\item Is it always the closure of the open ball? (No)
		\vskip 0.5ex
		Consider $\mathbb{R}$ which is equipped with a discrete metric. The open ball $B(0,1)$ under this metric space is the singleton $\{0\}$. So the closure is, $\overbar{B(0,1)}=\overbar{\{0\}} = \{0\}$. Meanwhile the closed ball with the same center and radius gives us, $B'(0,1)=\mathbb{R}$. Thus as this counterexample exists we can say that not always the closed ball becomes the closure of the open ball.
	\end{enumerate}
\end{proof}

\begin{problem}
	Let $A \subset B \subset X$. Show that $int(A) \subset int(B)$ and $\overbar{A} \subset \overbar{B}$. \\
\end{problem}

\begin{proof}
$ $ \newline
\vspace{-0.15in}
\begin{enumerate}
	\item $int(A) \subset int(B)$
	\vskip 0.5ex 
	Consider $x \in int(A)$ which implies that there exists $r>0$ such that $B(x,r)\subset A$. But as $A \subset B$, it is also true that $B(x,r)\subset B$. Thus $x \in int(B)$ holds.
	\item  $\overbar{A} \subset \overbar{B}$
	\vskip 0.5ex
	Consider $x \in \overbar{A}$, which implies that for any $r>0$, $B(x,r)\cap A \neq \phi$. Again as $A \subset B$, the previous statement implies that $B(x,r)\cap A \subset B(x,r)\cap B $. Thus $x \in \overbar{B}$ also holds.
\end{enumerate}
\end{proof}

\begin{problem}
	Let $A \subset Y \subset X$. Show the following:
	\begin{enumerate}[label=(\alph*)]
		\item 
		\item 
		\item 
		\item
	\end{enumerate}
\end{problem}

\end{document}