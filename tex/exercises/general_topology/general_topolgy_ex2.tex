\documentclass[paper=a4, fontsize=11pt]{scrartcl}

\usepackage[T1]{fontenc} 
\usepackage[english]{babel} 
\usepackage{amsmath}
\usepackage{amsfonts}
\usepackage{amsthm}
\usepackage{amssymb}
\usepackage{changepage}
\usepackage{titlesec}
\usepackage{sectsty} 
\sectionfont{\centering \normalfont \scshape}
\subsectionfont{\normalfont}
\subsubsectionfont{\normalfont}

\usepackage{fancyhdr} 
\pagestyle{fancyplain} 
\fancyhead{} 
\fancyfoot[L]{} 
\fancyfoot[R]{} 
\fancyfoot[C]{\thepage} 
\renewcommand{\headrulewidth}{0pt} 
\renewcommand{\footrulewidth}{0pt} 
\setlength{\headheight}{13.6pt} 

\usepackage{mathtools}
\usepackage{enumitem}
\newcommand{\subscript}[2]{$#1 _ #2$}

\newcommand{\horrule}[1]{\rule{\linewidth}{#1}} 
\newcommand{\ball}[2]{$B({#1};{#2})$}
\newcommand{\overbar}[1]{
	\mkern 1.5mu \overline{\mkern-1.5mu\raisebox{0pt}[\dimexpr\height+0.5mm\relax]{$#1$}\mkern-1.5mu}\mkern 1.5mu
}

\title{	
	\normalfont \normalsize 
	\textsc{Konkuk University Dept. Of Physics} \\ [25pt] %Konkuk University Dept. of Physics
	\horrule{1pt} \\[0.4cm] 
	\huge General Topology, Exercises 2 \\
	\vspace{0.1in}
	\Large 2019 Spring Semester
	\horrule{1pt} \\[0.4cm] 
}

\author{Youngwan Kim} 
\date{\normalsize\today} 
%\newtheorem{theorem}{Thm}
%\newtheorem{definition}{Def}
%\newtheorem{examples}{Examples}
%\newtheorem{example}{Ex}
%\newtheorem{lemma}{Lem}
%\newtheorem{corollary}{Cor}
\newtheorem*{remark}{Remark}
\newtheorem*{recall}{Recall}
\newtheorem{problem}{Problem}

\begin{document}
	
\maketitle	
\vspace{0.25in}

\begin{problem}
	Let $A=\{ (x,y) \in \mathbb{R}^2 : 1 < x^2+y^2 \leq 2 \}$. Find $int(A) , \overbar{A}$ and $\partial A$.\\
\end{problem}

\begin{proof}
$ $ \newline
\vspace{-0.15in}
\begin{enumerate}
	\item $int(A) = \{ (x,y) \in \mathbb{R}^2 : 1 < x^2+y^2 < 2 \}$ 
	\vskip 0.5ex 
	\item $\overbar{A}= \{ (x,y) \in \mathbb{R}^2 : 1 \leq x^2+y^2 \leq 2 \}$
	\vskip 0.5ex
	\item $\partial A = \{ (x,y) \in \mathbb{R}^2 : x^2+y^2 = 1 \text{ or } x^2+y^2 = 2 \}$
\end{enumerate}
\end{proof}

\begin{problem}
	Let $A=\{0\}\cup\{1/n : n \in \mathbb{N}\}$ be a subspace of $\mathbb{R}$. Find the limit points and the isolated points of $A$.\\
\end{problem}

\begin{proof}
$ $ \newline
\vspace{-0.15in}
\begin{enumerate}
	\item limit points : $\phi$
	\vskip 0.5ex
	
	\item isolated points : $A$
	\vskip 0.5ex
	
\end{enumerate}
\end{proof}

\begin{problem}
	Let $X$ be a finite metric space. Prove that $X$ has no limit points.\\
\end{problem}

\begin{proof}
	As $X$ is finite, we can denote its elements as $X=\{x_i\}$ for some $i\in I \subset \mathbb{N}$. Let $\epsilon>0$ be the smallest distance among all of the finite points of $X$, i.e, we will let $\epsilon = min\{ d(x_i,x_j) \}$. Now for $\forall x_i \in X$, consider the open ball $B(x_i,\epsilon)$. As for $\forall x_i$, there exists such $\epsilon$ that $B(x_i, \epsilon) \cap X = \{x_i\}$, every point of such finite metric space $X$ is an isolated point. As every point in $X$ is an isolated point, it implies that $X$ has no limit points. \\
\end{proof}

\begin{problem}
	Let $p,q\in X$. Show that there exists open subsets $U,V$ of $X$ such that $p\in U$ and $q \in V$ and $U \cap V = \phi$.\\
\end{problem}

\begin{proof}
	Let $\epsilon = \frac{1}{N} d(p,q) >0$ for sufficiently big $N>0$. Consider two open balls $B(p,\epsilon)$ and $B(q,\epsilon)$. As open balls are open, let each open balls as $U,V$. Then by definition it is obvious that $p\in U$ and $q \in V$. Now we claim that $U \cap V = \phi$. If there was such element in $U \cap V$, let it $z$, then by the trinagular identity of $d$, $d(p,q)\leq d(p,z) + d(z,q)$. But as $d(p,z)$ and $d(q,z)$ are smaller thatn $\epsilon < d(p,q)$ it leads to a contradiction that $d(p,q) < d(p,q)$. Thus such $z$ should not exist, and thus we conclude that for such $U,V$, $U \cap V = \phi$.\\
\end{proof}

\begin{problem}
	Consdier $\mathbb{Q}$ as a subspace of $\mathbb{R}$. Find the interior, the closure, the limit points and the isolated points of $\mathbb{Q}$.\\
\end{problem}

\begin{proof}
$ $ \newline
\vspace{-0.15in}
\begin{enumerate}
	\item $int(\mathbb{Q}) = \phi$ 
	\vskip 0.5ex
	For any $q\in \mathbb{Q}$, for any $r>0$, as the open ball $B(q,r)$ itself is consisted with real numbers, it never gets to be $B(q,r)\subset \mathbb{Q}$. Thus $int(\mathbb{Q})$ is the empty set.
	\item  $\overbar{\mathbb{Q}} = \mathbb{R}$
	\vskip 0.5ex
	For any $x \in \mathbb{R}$, there exists some $r>0$ such that $B(x,r)\cap \mathbb{Q} \neq \phi$ due to the property of $\mathbb{Q}$. To elaborate, for any $(a,b)\subset \mathbb{R}$ due to the density of rationals, there exists some $r \in \mathbb{R}$ such that $r \in (a,b)$. Also due to the Archimedean property, there exists some $n \in \mathbb{N}$ such that $a < a + \frac{b}{n} < b$. 
	\item  limit points : $\mathbb{R}$
	\vskip 0.5ex
	As the disjoint union of limit points and isolated points should be $\overbar{\mathbb{Q}}$ and as 
	\item  isolated points : $\phi$
	\vskip 0.5ex	 
	For any $q \in \mathbb{Q}$, 
\end{enumerate}
\end{proof}

\begin{problem}
	Prove that the set of irrational numbers $\mathbb{I}$ is dense in $\mathbb{R}$. \\
\end{problem}

\begin{proof}
	It suffices to show that $\bar{\mathbb{I}} = \mathbb{R}$ by the definition of dense subsets, which is same to show that for $\forall x \in \mathbb{R}$, there exists some $r'>0$ such that $B(x,r')\cap \mathbb{I} \neq \phi$. Then showing that for any $(a,b)\subset \mathbb{R}$, there exists some $k \in \mathbb{I}$ that is in $(a,b)$ would suffice it. Due to the density of rationals, there exists some $r \in \mathbb{R}$ such that $r \in (a,b)$. Using the Archimedean property, as $\frac{b-r}{2}>0$, there exists some $n \in \mathbb{N}$ such that $\frac{b-r}{2}> \frac{1}{n}$, which implies that $r+\frac{2}{n}<b$. As there exists some $k \in \mathbb{I}$ such that $a < k = r+\frac{\sqrt{2}}{n} < r+ \frac{2}{n} < b$ for any $(a,b) \in \mathbb{R}$, the closure of $\mathbb{I}$ is $\mathbb{R}$. \\
\end{proof}

\begin{problem}
	Let $A=\{0,1,2,3\}$ and $d_1$ be the usual metric on $A$ as a subspace of $\mathbb{R}$, and $d_2$ be the discrete metric on $A$. Are $d_1$ and $d_2$ equivalent? \\
\end{problem}

\begin{proof}
	First let us consider the open sets of $(A,d_2)$. As $d_2$ is a discrete metric, any singleton subset of $A$ is an open ball. That is because for any $x \in A$, $B(x,1)=\{x\}$. And using the fact that open balls are open and any arbitrary union of open sets are open, we can conclude that every subset of $A$ is actually open in $A$. Now let us consider the open sets of $(A,d_1)$. As $d_1$ is given as the usual metric of $\mathbb{R}$ and $A$ is a finite subset of such metric space, we know that every singleton subset of $A$ is closed in $A$. Thus using the fact that any finite intersection of closed sets are still closed, again we lead to the conclusion that every subset of $A$ is closed. Also using the fact that the complement of a closed set is open, we finally conclude that also in $(A,d_1)$ every subset of $A$ is also an open set. Thus we can say that $d_1$ and $d_2$ are equivalent. \\
\end{proof}

\begin{problem}
	Let $A=\{ (x,y) \in \mathbb{R}^2 | x^2 + y^2 < 1 \}$ be a subspace of $\mathbb{R}^2$. Show that it is not a complete space by giving an example of a Cauchy sequence in $A$ that doesn't converge in $A$.\\
\end{problem}

\begin{proof}
	Consider a sequence $\{ x _n \}_{n=1}^\infty$ where we define $x_n = (1-\frac{1}{n},0)$. As $x_n \to (1,0)$ it doesn't converge in $A$. Also $x_n$ is a Cauchy sequence as there exists $\epsilon>0$ and $N>0$ such that $d(x_n,x_m)=\left| \frac{1}{n} - \frac{1}{m} \right|$ for every $n,m \geq N$. Thus as there exists such $x_n$ is a Cauchy sequence in $A$ that doesn't converge in $A$, we can conclude that $A$ is not complete.  \\
\end{proof}

\end{document}