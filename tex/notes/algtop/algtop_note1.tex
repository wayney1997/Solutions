\documentclass[paper=a4, fontsize=11pt]{scrartcl}

\usepackage[T1]{fontenc} 
\usepackage[english]{babel} 
\usepackage{amsmath}
\usepackage{amsfonts}
\usepackage{amsthm}
\usepackage{amssymb}
\usepackage{changepage}
\usepackage{titlesec}
\usepackage{sectsty} 
\sectionfont{\centering \normalfont \scshape}
\subsectionfont{\normalfont}
\subsubsectionfont{\normalfont}

\usepackage{fancyhdr} 
\pagestyle{fancyplain} 
\fancyhead{} 
\fancyfoot[L]{} 
\fancyfoot[R]{} 
\fancyfoot[C]{\thepage} 
\renewcommand{\headrulewidth}{0pt} 
\renewcommand{\footrulewidth}{0pt} 
\setlength{\headheight}{13.6pt} 

\usepackage{enumitem}
\newcommand{\subscript}[2]{$#1 _ #2$}

\newcommand{\horrule}[1]{\rule{\linewidth}{#1}} 
\newcommand{\ball}[2]{$B({#1};{#2})$}
\newcommand{\overbar}[1]{
	\mkern 1.5mu \overline{\mkern-1.5mu\raisebox{0pt}[\dimexpr\height+0.5mm\relax]{$#1$}\mkern-1.5mu}\mkern 1.5mu
}

\title{	
	\normalfont \normalsize 
	\textsc{Konkuk University Dept. Of Physics} \\ [25pt] %Konkuk University Dept. of Physics
	\horrule{1pt} \\[0.4cm] 
	\huge Algebraic Topology \\
	\vspace{0.1in}
	\Large 2019 Spring Semester
	\horrule{1pt} \\[0.4cm] 
}

\author{Youngwan Kim} 
\date{\normalsize\today} 

\newtheorem{theorem}{Thm}
\newtheorem{definition}{Def}
%\newtheorem{examples}{Examples}
\newtheorem{example}{Ex}
\newtheorem{lemma}{Lem}
\newtheorem{corollary}{Cor}
\newtheorem*{remark}{Remark}
\newtheorem*{recall}{Recall}
\begin{document}
	
\maketitle	

\section{Introduction}

\vspace{0.15in}

If you ever had a course in topology before, you must have had assignments about whether some topological spaces are homeomorphic. For instance, the open interval $(0,1)$ in $\mathbb{R}$ is homeomorphic with $\mathbb{R}$. We can solve certain 'exercise problems' as they state us that they are 'homeomorphic' at first and all we have to do is just to find an adequate homeomorphism that maps one to another. But life is way much complicated than that, think of a situation where we have to actually check if two topological spaces are not homeomorphic. For instance, just with basic topological background it is very hard to show that $T^2$ and $S^2$ are not homeomorphic. \\

Algebraic topology gains certain significance regarding these kinds of situations where some topological spaces are given and we have to show that they are not homeomorphic. We implement algebraic methods to show such properties. From now on we will show that a topological space can be 'mapped' into a certain group, and if two topological spaces are homeomorphic those groups are also isomorphic. Algebraic topology can be broadly classified into two main topics : Homotopy theory and Homology theory, and the main goal of this course is to understand the fundamental group of topological spaces and show that $\pi_1 (S^1) = \mathbb{Z}$. We will not go through homology and cohomology theory.\\

\section{Homotopic Paths}

\vspace{0.15in}

We start our step by defining homotopic paths, which will later help us define fundamental groups. I skipped the documentation of the notes regarding basic group theory.\\

\begin{definition}[Paths]$ $\newline
	Let $X$ be a topological space. A \textbf{path} in $X$ is a continuous map $\gamma : [0,1] \to X$.\\
\end{definition}

There is nothing more than that. A path is simply a continuous map with the domain of $[0,1]$.\\

\begin{definition}[Homotopic Paths and Homotopy]
$ $\newline 
Let $X$ be a topological space and $a,b \in X$. Also let $\gamma_0 , \gamma_1$ be paths from $a$ to $b$, i.e, $\gamma_0(0)=\gamma_1(0)=a$ and $\gamma_0(1)=\gamma_1(1)=b$. Then $\gamma_0$ is \textbf{homotopic} to $\gamma_1$ with end points fixed if there exists a continuous map $F:[0,1]\times[0,1]\to X$ such that,
\vskip 0.5ex
\begin{enumerate}
	\item $F(s,0)=\gamma_0(s)$ for $s \in [0,1]$
	\item $F(s,1)=\gamma_1(s)$ for $s \in [0,1]$
	\item $F(0,t)=a$ for $t \in [0,1]$
	\item $F(1,t)=b$ for $t \in [0,1]$
\end{enumerate}
\vskip 0.5ex
where we call such $F$ a \textbf{homotopy}. We denote homotopic paths as $\gamma_0 \simeq \gamma_1$, $rel$ $\{0,1\}$.\\
\end{definition}

The third and fourth condition stated for such map being a homotopy is to emphasize that the end points are fixed and we even denote it as $rel$ $\{0,1\}$. Homotopy can be easily thought as some kind of continuous map deforming, or a 'movie' showing one path smoothly becoming the other one. \\

\begin{lemma}
	Homotopic relations are equivalence relations.\\
\end{lemma}

\begin{proof}
	We will just show that homotopic relations are transitive. Let $\gamma_0 \simeq \gamma_1$ by a homotopy $F$ and $\gamma_1 \simeq \gamma_2$ by a homotopy $G$ where all homotopy are $rel$ $\{0,1\}$. We will show that there exists a homotopy $H:[0,1]\times[0,1]\to X$ such that $\gamma_0 \simeq \gamma_2$, $rel$ $\{0,1\}$. We define such $H$ by,\\
	\begin{equation}\nonumber
		H(s,t) = 
		\begin{cases}
		F(s,2t) & t\in[0,\frac{1}{2})\\
		G(s,2t-1) & t\in[\frac{1}{2},1]
		\end{cases}
	\end{equation} 
	then we can show that,\\
	\begin{equation}\nonumber
		\begin{split}
		H(s,0) &= F(s,0) = \gamma_0(s)\\
		H(s,1) &= G(s,1) = \gamma_2(s)
		\end{split}
	\end{equation}\\
	and,
	\begin{equation}\nonumber
	\begin{split}
	H(0,t) &= a\\
	H(1,t) &= b
	\end{split}
	\end{equation}\\
	which suffices all of the four conditions to be a homotopy between $\gamma_0$ and $\gamma_2$.\\
\end{proof}

\begin{recall}
	By the gluing lemma, the continuity of $H$ is guranteed, but we won't elaborate it here.\\
\end{recall}

As we've seen that homotopy relation is actually an equivalence relation, it is natural to think of the equivalence classes that such relation makes.\\

\begin{definition}
	For a topological space $X$, where $\gamma$ is a path defined in $X$, we define the \textbf{homotopy class} of $\gamma$ as\\
	\begin{equation} \nonumber
		[\gamma] = \{\gamma ' : [0,1]\to X \text{ s.t }\ \gamma \simeq \gamma' \} 
	\end{equation}
\end{definition}
\vspace{0.15in}
Homotopy classes will be the basic notion for structuring fundamental groups.\\

\begin{remark}
	By definition, $\gamma_0 \simeq \gamma_1 \iff [\gamma_0]=[\gamma_1]$\\
\end{remark}

\begin{definition}
	For a topological space $X$, where $a \in X$, we denote the constant path $\gamma:[0,1]\to X$ which maps $\gamma(s)=a$ for $\forall s \in [0,1]$, by just $a$.\\
\end{definition}

\begin{lemma}[Independence of Reparametrization] $ $\newline
	Let $\gamma:[0,1]\to X$ be a path such that $\gamma(0)=a$ and $\gamma(1)=b$. Also let $\rho:[0,1]\to [0,1]$ be a continuous function such that $\rho(0)=0$ and $\rho(1)=1$. Then $[\gamma]=[\gamma \circ \rho]$.\\
\end{lemma}

\begin{proof}
	Define $F:[0,1]\times[0,1] \to X$ by $F(s,t) = \gamma((1-t)\cdot s + t \cdot \rho(s))$. Then $F(s,0)=\gamma(s)$, $F(s,1)=\gamma(\rho(s))$, $F(0,t)=a$, and $F(1,t)=b$. Thus for any $\rho$, $F$ suffices to be a homotopy between $\gamma$ and $\gamma \circ \rho$ thus $[\gamma]=[\gamma \circ \rho]$.\\
\end{proof}

\begin{definition}
	We define the collection of paths with fixed points and denote it as :
	\begin{equation}\nonumber
			P_X (a,b)=\{\gamma:[0,1]\to X :\gamma(0)=a,\gamma(1)=b\}
	\end{equation}
\end{definition}
\vspace{0.15in}

That collection looks lonely without a proper operation between paths, so let's define one.\\

\begin{definition}[Concatenation]  $ $ \newline
	Let $\alpha \in P_X(a,b)$ and $\beta \in P_X(b,c)$. Then we define the \textbf{concatenation} of $\alpha$ and $\beta$ and denote it as $\alpha\beta : [0,1] \to X$ by,
	\begin{equation}\nonumber
		\alpha\beta(s) = https://tex.stackexchange.com/questions/9065/large-braces-for-specifying-values-of-variables-by-condition
		\begin{cases}
		\alpha(2s)  & s \in [0,\frac{1}{2})\\
		\beta(2s-1) & s \in [\frac{1}{2},1]
		\end{cases} \\
	\end{equation}
	where $\alpha\beta \in P_X(a,c)$.\\
\end{definition}

\begin{lemma}[Homotopy Invariance]
	Let $\alpha_0,\alpha_1 \in P_X(a,b)$ such that $[\alpha_0]=[\alpha_1]$. Let $\beta_0,\beta_1 \in P_X(b,c)$ such that $[\beta_0]=[\beta_1]$. Then $[\alpha_0\beta_0]=[\alpha_1\beta_1]$.\\
\end{lemma}

\begin{proof}
	We need to show that there exists $H:[0,1]\times[0,1]\to X$
\end{proof}

\section{The Fundamental Group}

\section{Induced Homomorphisms}

\section{Covering Spaces}

\end{document}