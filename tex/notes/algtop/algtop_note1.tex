\documentclass[paper=a4, fontsize=11pt]{scrartcl}

\usepackage[T1]{fontenc} 
\usepackage[english]{babel} 
\usepackage{amsmath}
\usepackage{amsfonts}
\usepackage{amsthm}
\usepackage{amssymb}
\usepackage{changepage}
\usepackage{titlesec}
\usepackage{sectsty} 
\sectionfont{\centering \normalfont \scshape}
\subsectionfont{\normalfont}
\subsubsectionfont{\normalfont}

\usepackage{fancyhdr} 
\pagestyle{fancyplain} 
\fancyhead{} 
\fancyfoot[L]{} 
\fancyfoot[R]{} 
\fancyfoot[C]{\thepage} 
\renewcommand{\headrulewidth}{0pt} 
\renewcommand{\footrulewidth}{0pt} 
\setlength{\headheight}{13.6pt} 

\usepackage{enumitem}
\newcommand{\subscript}[2]{$#1 _ #2$}

\newcommand{\horrule}[1]{\rule{\linewidth}{#1}} 
\newcommand{\ball}[2]{$B({#1};{#2})$}
\newcommand{\overbar}[1]{
	\mkern 1.5mu \overline{\mkern-1.5mu\raisebox{0pt}[\dimexpr\height+0.5mm\relax]{$#1$}\mkern-1.5mu}\mkern 1.5mu
}

\title{	
	\normalfont \normalsize 
	\textsc{Konkuk University Dept. Of Physics} \\ [25pt] %Konkuk University Dept. of Physics
	\horrule{1pt} \\[0.4cm] 
	\huge Algebraic Topology \\
	\vspace{0.1in}
	\Large 2019 Spring Semester
	\horrule{1pt} \\[0.4cm] 
}

\author{Youngwan Kim} 
\date{\normalsize\today} 

\newtheorem{theorem}{Thm}
\newtheorem{definition}{Def}
%\newtheorem{examples}{Examples}
\newtheorem{example}{Ex}
\newtheorem{lemma}{Lem}
\newtheorem{corollary}{Cor}
\newtheorem*{remark}{Remark}
\newtheorem*{recall}{Recall}
\begin{document}
	
\maketitle	

\section{Introduction}

\vspace{0.15in}

If you ever had a course in topology before, you must have had assignments about whether some topological spaces are homeomorphic. For instance, the open interval $(0,1)$ in $\mathbb{R}$ is homeomorphic with $\mathbb{R}$. We can solve certain 'exercise problems' as they state us that they are 'homeomorphic' at first and all we have to do is just to find an adequate homeomorphism that maps one to another. But life is way much complicated than that, think of a situation where we have to actually check if two topological spaces are not homeomorphic. For instance, just with basic topological background it is very hard to show that $T^2$ and $S^2$ are not homeomorphic. \\

Algebraic topology gains certain significance regarding these kinds of situations where some topological spaces are given and we have to show that they are not homeomorphic. We implement algebraic methods to show such properties. From now on we will show that a topological space can be 'mapped' into a certain group, and if two topological spaces are homeomorphic those groups are also isomorphic. Algebraic topology can be broadly classified into two main topics : Homotopy theory and Homology theory, and the main goal of this course is to understand the fundamental group of topological spaces and show that $\pi_1 (S^1) = \mathbb{Z}$. We will not go through homology and cohomology theory.\\

\section{Homotopic Paths}

\vspace{0.15in}

We start our step by defining homotopic paths, which will later help us define fundamental groups. I skipped the documentation of the notes regarding basic group theory.\\

\begin{definition}[Paths]$ $\newline
	Let $X$ be a topological space. A \textbf{path} in $X$ is a continuous map $\gamma : [0,1] \to X$.\\
\end{definition}

There is nothing more than that. A path is simply a continuous map with the domain of $[0,1]$.\\

\begin{definition}[Homotopic Paths and Homotopy]
$ $\newline 
Let $X$ be a topological space and $a,b \in X$. Also let $\gamma_0 , \gamma_1$ be paths from $a$ to $b$, i.e, $\gamma_0(0)=\gamma_1(0)=a$ and $\gamma_0(1)=\gamma_1(1)=b$. Then $\gamma_0$ is \textbf{homotopic} to $\gamma_1$ with end points fixed if there exists a continuous map $F:[0,1]\times[0,1]\to X$ such that,
\vskip 0.5ex
\begin{enumerate}
	\item $F(s,0)=\gamma_0(s)$ for $s \in [0,1]$
	\item $F(s,1)=\gamma_1(s)$ for $s \in [0,1]$
	\item $F(0,t)=a$ for $t \in [0,1]$
	\item $F(1,t)=b$ for $t \in [0,1]$
\end{enumerate}
\vskip 0.5ex
where we call such $F$ a \textbf{homotopy}. We denote homotopic paths as $\gamma_0 \simeq \gamma_1$, $rel$ $\{0,1\}$.\\
\end{definition}

The third and fourth condition stated for such map being a homotopy is to emphasize that the end points are fixed and we even denote it as $rel$ $\{0,1\}$. Homotopy can be easily thought as some kind of continuous map deforming, or a 'movie' showing one path smoothly becoming the other one. \\

\begin{lemma}
	Homotopic relations are equivalence relations.\\
\end{lemma}

\begin{proof}
	We will just show that homotopic relations are transitive. Let $\gamma_0 \simeq \gamma_1$ by a homotopy $F$ and $\gamma_1 \simeq \gamma_2$ by a homotopy $G$ where all homotopy are $rel$ $\{0,1\}$. We will show that there exists a homotopy $H:[0,1]\times[0,1]\to X$ such that $\gamma_0 \simeq \gamma_2$, $rel$ $\{0,1\}$. We define such $H$ by,\\
	\begin{equation}\nonumber
		H(s,t) = 
		\begin{cases}
		F(s,2t) & t\in[0,\frac{1}{2})\\
		G(s,2t-1) & t\in[\frac{1}{2},1]
		\end{cases}
	\end{equation} 
	then we can show that,\\
	\begin{equation}\nonumber
		\begin{split}
		H(s,0) &= F(s,0) = \gamma_0(s)\\
		H(s,1) &= G(s,1) = \gamma_2(s)
		\end{split}
	\end{equation}\\
	and,
	\begin{equation}\nonumber
	\begin{split}
	H(0,t) &= a\\
	H(1,t) &= b
	\end{split}
	\end{equation}\\
	which suffices all of the four conditions to be a homotopy between $\gamma_0$ and $\gamma_2$.\\
\end{proof}

\begin{recall}
	By the gluing lemma, the continuity of $H$ is guranteed, but we won't elaborate it here.\\
\end{recall}

As we've seen that homotopy relation is actually an equivalence relation, it is natural to think of the equivalence classes that such relation makes.\\

\begin{definition}
	For a topological space $X$, where $\gamma$ is a path defined in $X$, we define the \textbf{homotopy class} of $\gamma$ as\\
	\begin{equation} \nonumber
		[\gamma] = \{\gamma ' : [0,1]\to X \text{ s.t }\ \gamma \simeq \gamma' \} 
	\end{equation}
\end{definition}
\vspace{0.15in}
Homotopy classes will be the basic notion for structuring fundamental groups.\\

\begin{remark}
	By definition, $\gamma_0 \simeq \gamma_1 \iff [\gamma_0]=[\gamma_1]$\\
\end{remark}

\begin{definition}
	For a topological space $X$, where $a \in X$, we denote the constant path $\gamma:[0,1]\to X$ which maps $\gamma(s)=a$ for $\forall s \in [0,1]$, by just $a$.\\
\end{definition}

\begin{lemma}[Independence of Reparametrization] $ $\newline
	Let $\gamma:[0,1]\to X$ be a path such that $\gamma(0)=a$ and $\gamma(1)=b$. Also let $\rho:[0,1]\to [0,1]$ be a continuous function such that $\rho(0)=0$ and $\rho(1)=1$. Then $[\gamma]=[\gamma \circ \rho]$.\\
\end{lemma}

\begin{proof}
	Define $F:[0,1]\times[0,1] \to X$ by $F(s,t) = \gamma((1-t)\cdot s + t \cdot \rho(s))$. Then $F(s,0)=\gamma(s)$, $F(s,1)=\gamma(\rho(s))$, $F(0,t)=a$, and $F(1,t)=b$. Thus for any $\rho$, $F$ suffices to be a homotopy between $\gamma$ and $\gamma \circ \rho$ thus $[\gamma]=[\gamma \circ \rho]$.\\
\end{proof}

\begin{definition}
	We define the collection of paths with fixed points and denote it as :
	\begin{equation}\nonumber
			P_X (a,b)=\{\gamma:[0,1]\to X :\gamma(0)=a,\gamma(1)=b\}
	\end{equation}
\end{definition}
\vspace{0.15in}

That collection looks lonely without a proper operation between paths, so let's define one.\\

\begin{definition}[Concatenation]  $ $ \newline
	Let $\alpha \in P_X(a,b)$ and $\beta \in P_X(b,c)$. Then we define the \textbf{concatenation} of $\alpha$ and $\beta$ and denote it as $\alpha\beta : [0,1] \to X$ by,
	\begin{equation}\nonumber
		\alpha\beta(s) = 
		\begin{cases}
		\alpha(2s)  & s \in [0,\frac{1}{2})\\
		\beta(2s-1) & s \in [\frac{1}{2},1]
		\end{cases} \\
	\end{equation}
	where $\alpha\beta \in P_X(a,c)$.\\
\end{definition}

\begin{lemma}[Homotopy Invariance]
	Let $\alpha_0,\alpha_1 \in P_X(a,b)$ such that $[\alpha_0]=[\alpha_1]$. Let $\beta_0,\beta_1 \in P_X(b,c)$ such that $[\beta_0]=[\beta_1]$. Then $[\alpha_0\beta_0]=[\alpha_1\beta_1]$.\\
\end{lemma}

\begin{proof}
	We need to show that there exists $H:[0,1]\times[0,1]\to X$ such that $\alpha_0\beta_0 \simeq \alpha_1\beta_1$, $rel$ $\{0,1\}$. Let $F,G$ each be the homotopy such that $\alpha_0 \simeq \alpha_1$ and $\beta_0 \simeq \beta_1$ all $rel$ $\{0,1\}$. If we define $H(s,t)$ by, \\
	\begin{equation}\nonumber
		H(s,t) =
		\begin{cases}
		 F(2s,t) & s\in [0,\frac{1}{2})\\
		 G(2s-1,t) & s \in [\frac{1}{2},1]
		\end{cases}
	\end{equation}\\
	 we can easily check that this $H$ suffices to be a homotopy between $\alpha_0 \beta_0$ and $\alpha_1\beta_1$.\\
\end{proof}

\begin{remark}
	Due to homotopy invariance, we can define $[\alpha][\beta]=[\alpha\beta]$.\\
\end{remark}

\begin{lemma}
	Let $X$ be a topological space. For $\alpha\in P_X(a,b)$, $\beta \in P_X(b,c)$, and $\gamma \in P_X(c,d)$, \\
	\begin{equation}\nonumber
		([\alpha][\beta])[\gamma] = [\alpha]([\beta][\gamma]).
	\end{equation}
\end{lemma}
\vspace{0.15in}
This lemma states that such operation over homotopy classes is associative, and we also wish there is an identity like object for this class. We will later show that the homotopy class of constant paths actually acts as an identity element for such operation.\\
\begin{remark}
	$(\alpha\beta)\gamma$ and $\alpha(\beta\gamma)$ are different paths, but homotopic thanks to \textbf{Lem 2}. Do not regard them as identical 'paths'. \\
\end{remark}

\begin{lemma}
	Let $\alpha \in P_X(a,b)$. Then $[a][\alpha]=[\alpha]$ and $[\alpha][b]=[\alpha]$ where,
	\begin{equation}\nonumber
		a\alpha(s) =
		\begin{cases}
		\alpha(2s) & \\
		\alpha(2s-1)
		\end{cases}
	\end{equation}
\end{lemma}

\vspace{0.1in}

\begin{definition}
	Let $\alpha\in P_X(a,b)$. Then we define the inverse path of $\alpha$ as $\alpha^{-1}:[0,1]\to X$ which maps $\alpha^{-1}(s) = \alpha(1-s)$. \\
\end{definition}

\begin{remark}
	We can see that for $\alpha\in P_X(a,b)$ the inverse of it $\alpha^{-1}\in P_X(b,a)$.\\
\end{remark}

\begin{lemma}
	Let $\alpha\in P_X(a,b)$. Then $[\alpha][\alpha^{-1}]=[a]$ and $[\alpha^{-1}][\alpha]=[b]$.\\
\end{lemma}

\begin{proof}
	We will just show the case that $[\alpha][\alpha^{-1}]=[a]$ by taking a homotopy $F$ as,
	\begin{equation}\nonumber
		F(s,t) =
		\begin{cases}
		\alpha(2s) & s \in [0,\frac{t}{2}) \\
		\alpha(t) &  s \in [\frac{t}{2}, 1-\frac{t}{2})\\
		\alpha(2-2s) & s \in [1-\frac{t}{2},1]
		\end{cases}
	\end{equation}
	We can then check that $F$ becomes a homotopy between $a$ and $\alpha\alpha^{-1}$.\\
\end{proof}

\begin{lemma}
	Let $\alpha_0,\alpha_1 \in P_X(a,b)$. If $\alpha_0 \simeq \alpha_1$, $rel$ $\{0,1\}$ then $\alpha_0^{-1}\simeq \alpha_1^{-1}$, $rel$ $\{0,1\}$.\\
\end{lemma}

\begin{proof}
	Let $F$ be the homotopy between $\alpha_0$ and $\alpha_1$. We define $G(s,t)=F(1-s,t)$ and then this becomes a homotopy bewtween  $\alpha_0^{-1}$ and $\alpha_1^{-1}$. \\
\end{proof}

\begin{definition}
	Let $\alpha \in P_X(a,b)$, then we define the inverse homotopy class of $\alpha$ by $[\alpha]^{-1} = [\alpha^{-1}]$.\\
\end{definition}
\vspace{0.15in}

%----------------------------------------

\section{The Fundamental Group}
\vspace{0.15in}
Throughout the previous section, we defined many concepts that will be essential for us to define the fundamental group of a topological space. Now using the concepts from the previous section, now we define the concept of fundamental group.\\

\begin{definition}
	For a topological space $X$ with $b \in X$, we define a set named $\pi_1(X,b)$ of $X$ by
	\begin{equation}\nonumber
		\begin{split}
		\pi_1(x,b) &= L(X,b)/\simeq\\
		&= \{[\alpha] \text{ }|\text{ } \alpha : [0,1] \to X \text{, s.t } \alpha(0)=\alpha(1)=b\}
		\end{split}
	\end{equation}
	where $L(X,b)$ is the set of loops in $X$ having $b \in X$ as a base point.\\
\end{definition}

\begin{theorem}
	If we equip the set $\pi_1(x,b)$ with the operation $[\alpha][\beta]=[\alpha\beta]$, $\pi_1(X,b)$ becomes a group.\\
\end{theorem}

\begin{recall}
	A topological space $X$ is \textbf{path connected} if for $\forall a,b \in X$, there exists a path $\gamma:[0,1]\to X$ such that $\gamma(0)=a$ and $\gamma(1)=b$.\\
\end{recall}

\begin{theorem}
	Let $X$ be a path connected topological space. Then $\pi_1(X,b) \simeq \pi_1(X,c)$ for any $b,c\in X$.\\
\end{theorem}

\begin{proof}
	Define $\gamma:[0,1]\to X$ be a path such that $\gamma(0)=c$ and $\gamma(1)=b$. Existence of such path is guranteed due to the fact that $X$ is path connected. Then define a mapping $\phi : \pi_1(X,b) \to \pi_1(X,c)$, such that 
	\begin{equation}\nonumber
		\phi([\alpha])=[\gamma][\alpha][\gamma]^{-1} = [\gamma\alpha\gamma^{-1}]
	\end{equation}
	We will then show that such $\phi$ is an isomorphism between $\pi_1(X,b)$ and $\pi_1(X,c)$.
	\begin{enumerate}[label=\arabic*)]
		\item $\phi$ is injective.
		\vskip 0.5ex
		Suppose $\phi([\alpha])$=$\phi([\beta]) \implies [\gamma][\alpha][\gamma]^{-1} = [\gamma][\beta][\gamma]^{-1}$ for any $[a]$ and $[\beta]$. Then,
		\begin{equation}\nonumber
			\begin{split}
			[\gamma][\alpha][\gamma]^{-1} &= [\gamma][\beta][\gamma]^{-1} \\
			[\gamma]^{-1}[\gamma][\alpha][\gamma]^{-1}[\gamma] &= [\gamma]^{-1}[\gamma][\beta][\gamma]^{-1}[\gamma] \\
			[b][\alpha][c] &= [b][\beta][c] \\
			[\alpha] &= [\beta]
			\end{split}
		\end{equation} 
		\item $\phi$ is surjective.
		\vskip 0.5ex
		Let arbitrary $[\alpha']\in\pi_1(X,c)$. Consider $\phi([\gamma^{-1}\alpha'\gamma])$, 
		\begin{equation}\nonumber
			\begin{split}
			\phi([\gamma^{-1}\alpha'\gamma]) &= [\gamma][\gamma^{-1}\alpha'\gamma][\gamma^{-1}]\\
			&= [c][\alpha'][c] = [\alpha']
			\end{split}
		\end{equation}
		\item $\phi$ is a homomorphism.
		\vskip 0.5ex
		For any $[\alpha],[\beta]\in\pi_1(X,b)$, 
		\begin{equation}\nonumber
		\begin{split}
		\phi([\alpha][\beta]) &= [\gamma][\alpha][\beta][\gamma]^{-1}\\
		&= [\gamma][\alpha][b][\beta][\gamma]^{-1} \\
		&= [\gamma][\alpha][\gamma]^{-1}[\gamma][\beta][\gamma]^{-1} \\
		&= \phi([\alpha])\phi([\beta])
		\end{split}
		\end{equation}
	\end{enumerate}
	As $\phi$ is a bijective homomorphism between $\pi_1(X,b)$ and $\pi_1(X,c)$, it is an isomorphism.
\end{proof}

\begin{definition}
	As \textbf{Thm 2} states that for path connected topological spaces $\pi_1$ is independent of the choice of base point, we define the \textbf{fundamental group} of a path connected topological space as the isomorphic type of $\pi_1(X,b)$ and denote it as $\pi_1(X)$.\\
\end{definition}

\begin{definition}
	A topolgical space $X$ is \textbf{simply connected} if $X$ is path connected and the fundamental group is trivial, i.e, $\pi_1(X)=0$. \\
\end{definition}

\begin{example} Here are some examples of simply connected and not simply connected spaces.
	\begin{itemize}
		\item Simply connected : $\pi_1(\mathbb{R}^n)=0$, $\pi_1(D^n)=0$, $\pi_1(S^n)=0$ (for $n>1$) ...
		\item Not simply connected : $\pi_1(S^1)=\mathbb{Z}$, $\pi_1(\mathbb{R}\setminus\{0\})=\mathbb{Z}$ ...\\
	\end{itemize}
\end{example}

\begin{example}
	We will explicitly show that $\mathbb{R}^2$ is simply connected.\\ \vskip 0.2ex
	Let the base point $b=(0,0)\in\mathbb{R}^2$. Also let $\alpha:(0,1]\to\mathbb{R}^2$ be a loop beased at $b$. Define $F:[0,1]\times[0,1]\to \mathbb{R}^2$ by\\
	\begin{equation}\nonumber
		F(s,t) = t \cdot b + (1-t) \cdot \alpha
	\end{equation}\\
	which eventually becomes a homotopy such that $\alpha \simeq b$ for any loop based at $b$. This implies that $\pi_1(X,b)=\{[b]\}$, which is the trivial group.\\
\end{example}

From the above example, we took advantage of the convex property of $\mathbb{R}^2$ when we constructed the homotopy $F$. This means that we could apply the same method of proof to any convex spaces. \\

\begin{definition}
	A subset $S$ of $\mathbb{R}^n$ is \textbf{convex} if for $\forall a,b \in S$, the line segment from $a$ to $b$ is contained in $S$, i.e, $\forall t \in [0,1]$ the line segment $t\cdot a + (1-t)\cdot b \in S$.\\
\end{definition} 

\begin{example}
	Some basic examples of convex and non convex subsets of $\mathbb{R}^n$.
	\begin{itemize}
		\item $\mathbb{R}^n$ itself is convex.
		\item $D^n=\{x\in \mathbb{R}^n : |x|\leq 1 \}$ is convex.
		\item $S^1=\{x\in \mathbb{R}^2 : |x|=1\}$ is not convex.\\
	\end{itemize}
\end{example}

Then as we expect, does any convex subset becomes a simply connected space? The next theorem states that it is true.\\

\begin{theorem}
	A convex subset $S$ of $\mathbb{R}^n$ is simply connected.\\
\end{theorem}

\begin{proof}
	Let $b\in S$, and $\alpha:[0,1]\to S$ be a loop based at $b$. Define $F:[0,1]\times[0,1] \to S$ as $F(s,t)=t\cdot b + (1-t)\cdot \alpha$ which becomes a homotopy such that $\alpha \simeq b$. The existence of such homotopy is guranteed as $S$ is a convex subset. As for any loop based at $b$ becomes homotopic with $b$, $\pi_1(S)=0$ thus $S$ is simply connected.\\
\end{proof}

\begin{remark}
	If a subset of $\mathbb{R}^n$ is convex, it is also path connected.\\
\end{remark}

\begin{definition}
	A subset $S$ of $\mathbb{R}^n$ is \textbf{star shaped} if there exists some $b\in S$ such that for each $x \in S$ the line between $b$ and $x$ is contained in $S$, i.e, for $\forall t\in[0,1]$ the line segment $t\cdot a + (1-t)\cdot b \in S$.\\
\end{definition}

\begin{theorem}
	A star shaped subset is simply connected.
\end{theorem}

\begin{proof}
	The proof is skipped.\\
\end{proof}

\begin{remark}
	convex $\implies$ star shaped $\implies$ simply connected\\
\end{remark}

The homotopy concept can be actually extended to a more general one, as our definition of homotopy can be regarded a continuous function mapping one path which maps $[0,1]\to X$ to another path. Instead of $[0,1]$ we can similarly define homotopy with any topological space.\\ 

\begin{definition}
	Let $X,Y$ be topological spaces. Let $A\subset Y$ and $f_0,f_1:Y \to X$ be maps such that $f_0(a)=f_1(a)$ for $\forall a \in A$. Then $f_0$ is \textbf{homotopic} to $f_1$ \textbf{relative to} $A$ if there exists $F:Y\times[0,1]\to X$ such that $F(y,0)=f_0(y)$ and $F(y,1)=f_1(y)$ for $y\in Y$ and $F(a,t)=f_0(a)=f_1(a)$ and $F(a,t)=f_0(a)=f_1(a)$ for $\forall a \in A$. We denote such relation as $f_0 \simeq f_1$, $rel$ $A$.\\
\end{definition}

\begin{definition}
	Let $X$ be a topological space where $x_0 \in X$. $X$ is \textbf{contractible to} $x_0 \in X$ with $x_0$ held fixed if there exists a map $F:X\times[0,1]\to X$ such that $F(x,0)=x$ and $F(x,1)=x_0$ for $\forall x \in X$ and $F(x_0,t)=x_0$ for $\forall t \in [0,1]$.\\
\end{definition}

\begin{theorem}
	If $X$ is star shaped, then $X$ is contractible.
\end{theorem}

\begin{theorem}
	If $X$ is contractible, then $\pi_1(X)=0$.\\
\end{theorem}

\begin{proof}
	Let $\alpha:[0,1]\to X$ be a path such that $\alpha(0)=\alpha(1)=x_0$. Let $F:X\times[0,1]\to X$ be a map such that $F(x,0)=x$, $F(x,1)=x_0$, $F(x_0,t)=x_0$ as $X$ is contractible. Now let us define $G:[0,1]\times[0,1] \to X$ such that $G(s,t)=F(\alpha(s),t)$. Then, 
	\begin{equation}\nonumber
		\begin{cases}
		G(s,0)=F(\alpha(s),0)=\alpha(s) & \\
		G(s,1)=F(\alpha(s),1)=x_0 & \\
		G(0,t)=F(x_0,t)=x_0 & \\
		G(1,t)= F(x_0,t)=x_0
		\end{cases}
	\end{equation}\\
	where we can check that $G$ becomes a homotopy between $\alpha$ and $x_0$, $rel$ $\{0,1\}$.\\
\end{proof}

Consider $S^n$, we know that when $n=1$ it isn't contractible, and we can show this both implicitly by taking arbitrary points in $S^1$ and showing that not all of the line segments are in $S^1$. 

\begin{table}[h!]
	\centering
	\begin{tabular}{|c|c|c|}
		\hline
		$X$   & contractible? & $\pi_1(X)$   \\ \hline
		$S^1$ & No            & $\mathbb{Z}$ \\ \hline
		$S^n$ & ??            & $0$          \\ \hline
	\end{tabular}\\
\end{table}

Only with homotopy theory we can't show whether $S^n$ are contractible or not for $n>1$. This question can be solved with homology theory but that is currently out of our scope.  
\vspace{0.15in}

%----------------------------------

\section{Induced Homomorphisms}
\vspace{0.15in}
We can think of obtaining a fundamental group of a certain topological space as some kind of correspondence, or a map sending a topological space to a group. The main question we deal with in this section is about whether there would also be a homomorphism between the fundamental groups induced by homeomorphisms between the topological spaces. \\

\begin{definition}
 Let $X,Y$ be topological spaces where $f:X\to Y$ is a homeomorphism between them such that for $b \in X$ and $c \in Y$, $f(b)=c$. Then we define the \textbf{induced homomorphism} of $f$ as $f_* : \pi_1(X,b) \to \pi_1(X,c)$ which maps $f_*([\alpha])=[f \circ \alpha]$.\\
\end{definition}

We can first define such induced homomorphisms, and the next lemma states that such homomorphisms are well defined.\\

\begin{lemma}
	Let $X,Y$ be a topological space with $f:X\to Y$. For $[\alpha_0],[\alpha_1]\in\pi_1(X,b)$ such that $[\alpha_0]=[\alpha_1]$,  $[f \circ \alpha_0]=[f\circ \alpha_1]$ in $\pi_1(Y,c)$.\\	
\end{lemma}

\begin{proof}
	Since $[\alpha_0]=[\alpha_1]$, there exists a homotopy between them and let it $F$. Define $G:[0,1]\times[0,1]\to Y$ by $G(s,t)=(f \circ F)(s,t)$. We can check that this becomes a homotopy between $f \circ \alpha_0$ and $f \circ \alpha_1$. \\
\end{proof}

\begin{theorem}
	Let $X,Y,Z$ be a topological space.
	\begin{enumerate}
		\item If $id : X \to X$ is the identity map, then $(id)_* : \pi_1(X,b) \to \pi_1(x,b)$ is the identity homomorphism of $\pi_1(X,b)$.
		\item If $f:(X,b)\to(Y,c)$ and $g:(Y,c)\to(Z,d)$, then $(g \circ f)_* = g_* \circ f_*$.\\
	\end{enumerate}
\end{theorem}

\begin{proof}
$ $ \newline
\vspace{-0.15in}
	\begin{enumerate}[label=\arabic*)]
		\item For $\forall [\alpha] \in \pi_1(X,b)$, by definition $(id_*)([\alpha])=[id \circ \alpha]=[\alpha]$. Thus $(id)_*$ is the identity homomorphism of $\pi_1(X,b)$.
		\item For $\forall [\alpha] \in \pi_1(X,b)$, the 
		\begin{equation}\nonumber
			\begin{split}
				(g \circ f)_* ([\alpha]) &= [(g \circ f) \circ \alpha]\\
				&= [g \circ (f \circ \alpha)] \\
				&= g_* ([f \circ \alpha]) \\
				&= g_* (f_* ([\alpha])) =  g_* \circ f_* ([\alpha])
			\end{split}
		\end{equation}
	\end{enumerate}
\end{proof}

\begin{theorem}
	If $f:X\to Y$ is a homeomorphism, then $f_*:\pi_1(X) \to \pi_1(Y)$ is an isomorphism.\\
\end{theorem}

\begin{proof}
	Since $f:X\to Y$ is a homeomorphism there exists a continuous map $f^{-1}:Y\to X$ such that $f^{-1} \circ f = id_X $ and $f \circ f^{-1} = id_Y$. Then,\\
	\begin{equation}\nonumber
		\begin{cases}
		(f^{-1} \circ f)_*= (id_X)_* & : \pi_1(X) \to \pi_1(X)\\
		(f \circ f^{-1} )_* = (id_Y)_* & : \pi_1(Y) \to \pi_1(Y)
		\end{cases}
	\end{equation}
	Also, 
	\begin{equation}\nonumber
	\begin{cases}
	(f^{-1} \circ f)_*= (f^{-1})_* \circ f_* & \\
	(f \circ f^{-1} )_* = f_* \circ (f^{-1})_* &
	\end{cases}
	\end{equation}\\
	and $(id_X)_* = id_{\pi_1(X)}$ and $(id_Y)_* = id_{\pi_1(Y)}$,
	\begin{equation}\nonumber
		\begin{cases}
			(f^{-1})_* \circ f_* = id_{\pi_1(X)}& \\
			f_* \circ (f^{-1})_* = id_{\pi_1(Y)}&
		\end{cases}
	\end{equation}
	As there exists such $(f^{-1})_*$, we conclude that such $f_*$ is a bijective map. Also, as $f_*$ is an induced homomorphism, $f_*$ is a bijective homomorphism, which is an isomorphism. Thus $f_*$ is an isomorphism.\\
\end{proof}

\begin{corollary}
	$X \cong Y \implies \pi_1(X) \simeq \pi_1(Y)$ which is, $\pi_1(X) \not\simeq \pi_1(Y) \implies X \not\cong Y$\\
\end{corollary}

This is a very useful corollary, because it directly tells us how to check if two topological spaces are non homeomorphic by comparing if their fundamental groups are isomorphic. This is how algebraic topology converts a topological problem into a purely algebraic situation.\\

\begin{corollary}
	If $X \cong Y$ and $X$ is simply connected, then $Y$ is also simply connected.\\
\end{corollary}

\begin{definition}
	Let $A \subseteq X$ for a topological space $X$.
	\begin{enumerate}
		\item A map $r:X \to A$ is a \textbf{retraction} of $X$ into $A$ if $r(a)=a$ for $\forall a \in A$
		\item For a retraction $r:X\to A$, we call $A$ the \textbf{retract} of $X$.\\
	\end{enumerate}
\end{definition}

\begin{example}
	Let $X=\mathbb{R}^2 \setminus \{0\}$, and $A=\{x \in \mathbb{R}^2 : |x|=1\}$. Define $r:X\to A$ as $r(x)=\frac{x}{|x|}$. Then as for $\forall a \in A$, $r(a)=a$ so $r$ becomes a retraction of $X$ into $A$ and $A$ becomes a retract of $X$. However, if we consider this map in $\mathbb{R}^2$ instead of $X$ it doesn't become a retraction and it is not easy to show directly.\\
\end{example}

\begin{theorem}
	If $A$ is a retract of $X$ and $X$ is simply connected, then $A$ is also simply connected.\\
\end{theorem}

\begin{proof}
	left as an exercise (check assignment sheet)\\
\end{proof}

Until now we defined and stated what are fundamental groups are, and the relation between them. But we haven't yet introduced how to calculate fundamental groups of a given topological space. There are two methods of calculating the fundamental group : one is to use the Seifert Van Kampen theorem, and the other is to deform the given topological space. The above theorem can be related to the second method, which was kind of a brief introduction. Now we also give a brief introduction  about the first method of calculating the fundamental group by stating a simple case of the Seifert Van Kampen theorem. \\

\begin{theorem}Let $U,V \subseteq X$ for a topological space $X$. Suppose that,
	\begin{enumerate}
		\item $U,V $ are open in $X$ and both are simply connected.
		\item $U \cap V = \phi$ and $U \cup V = X$.
		\item $U \cap V$ is path connected. 
	\end{enumerate}	
	Then $\pi_1(X)=0$, i.e, $X$ is also simply connected.
\end{theorem}

In order to prove this theorem, we need to introduce a lemma we learned from general topology. \\

\begin{lemma}
	
\end{lemma}

\section{Covering Spaces}

\end{document}