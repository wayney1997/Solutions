\documentclass[paper=a4, fontsize=11pt]{scrartcl}

%\usepackage[T1]{fontenc} 
\usepackage[english]{babel} 
\usepackage{amsmath}
\usepackage{amsfonts}
\usepackage{amsthm}
\usepackage{amssymb}
\usepackage{changepage}
\usepackage{titlesec}
\usepackage{sectsty} 
\sectionfont{\centering \normalfont \scshape}
\subsectionfont{\normalfont}
\subsubsectionfont{\normalfont}

\usepackage{fancyhdr} 
\pagestyle{fancyplain} 
\fancyhead{} 
\fancyfoot[L]{} 
\fancyfoot[R]{} 
\fancyfoot[C]{\thepage} 
\renewcommand{\headrulewidth}{0pt} 
\renewcommand{\footrulewidth}{0pt} 
\setlength{\headheight}{13.6pt} 

\usepackage{enumitem}
\newcommand{\subscript}[2]{$#1 _ #2$}

\newcommand{\T}{\mathcal{T}}
\newcommand{\nextline}{$ $ \newline \vspace{-0.15in}}
\newcommand{\horrule}[1]{\rule{\linewidth}{#1}} 
\newcommand{\ball}[2]{$B({#1};{#2})$}
\newcommand{\overbar}[1]{
	\mkern 1.5mu \overline{\mkern-1.5mu\raisebox{0pt}[\dimexpr\height+0.5mm\relax]{$#1$}\mkern-1.5mu}\mkern 1.5mu
}

\title{	
	\normalfont \normalsize 
	\textsc{Konkuk University Dept. Of Physics} \\ [25pt] %Konkuk University Dept. of Physics
	\horrule{1pt} \\[0.4cm] 
	\huge General Topology \\
	\vspace{0.1in}
	\Large 2019 Spring Semester
	\horrule{1pt} \\[0.4cm] 
}

\author{Youngwan Kim} 
\date{\normalsize\today} 

\newtheorem{theorem}{Thm}
\newtheorem{definition}{Def}
%\newtheorem{examples}{Examples}
\newtheorem{example}{Ex}
\newtheorem{lemma}{Lem}
\newtheorem*{remark}{Remark}
\newtheorem*{recall}{Recall}
\begin{document}
	
\maketitle	

\section{Metric Spaces} 
\vspace{2.5ex}

\subsection{Open and Closed Sets}

\vspace{2.5ex}

\begin{definition}
	A \textbf{metric} $d$ defined on a set $X$ is a mapping $d:X \times X \to \mathbb{R}$ which has the following properties.
	\begin{enumerate}[label=(\subscript{m}{{\arabic*}})]
		\item $\forall x,y \in X : d(x,y) \geq 0$ , $d(x,y)=0 \iff x=y$
		\item $\forall x,y \in X: d(x,y)=d(y,x)$
		\item $\forall x,y,z \in X : d(x,y)+d(y,z) \geq d(x,z)$\\
	\end{enumerate}
\end{definition}
If such map is well defined on a certain set $X$, we can now introduce the notion of metric spaces, which is merely a set equipped with a well defined metric.
\\
\begin{definition}
	A \textbf{metric space} $(X,d)$ is a set $X$ equipped with a metric $d:X\times X \to \mathbb{R}$.\\
\end{definition}
\begin{example} 
	These are some simple examples of metric spaces.
	\begin{enumerate}[label=\arabic*)]
		\item $(\mathbb{R}^n,d)$ where $d(x,y)=$
		\item $(X,d)$ where $d(x,y)=0$ if $x=y$ and $d(x,y)=1$ if $x\neq y$ \textbf{(discrete metric)}
	\end{enumerate}
\end{example}

Now it is natural to think of how the metric would be affected under the subsets, and the following theorem states that the restriction of such metric is still a metric on the subsets. \\
\begin{theorem}
	Let $(X,d)$ be a metric space and let $Y\subset X$. Then $d\restriction_{Y\times Y} : Y \times Y \to \mathbb{R}$ is a metric on $Y$.
\end{theorem}
\begin{proof}
	left as an exercise
\end{proof}
\vspace{2.5ex}
\begin{definition}
	Let $(X,d)$ be a metric space and $x\in X$, $r \in \mathbb{R}$ where $r>0$. The \textbf{open ball} \ball{x}{r} is defined as $\{y\in X : d(x,y)<r\}$ where $x$ is called the \textbf{center} of \ball{x}{r}, and $r$ is called the \textbf{radius} of \ball{x}{r}.\\
\end{definition}

\begin{example}
	Some basic examples of open balls on different metric spaces.
	\begin{enumerate}[label=\arabic*)]
		\item On $(\mathbb{R},d)$ where $d$ is the usual metric, \ball{x}{r}$=(x-r,x+r)$.
		\item On $(\mathbb{R},d)$ where $d$ is the discrete metric, \ball{x}{r}$=\{x\}$  if $r\in(0,1]$ and  \ball{x}{r}$=\mathbb{R}$  if $r>1$.\\
	\end{enumerate}
\end{example}

\begin{lemma}
	Let $(X,d)$ be a metric space, where $x\in X$ and $r>0$. Then,
	\begin{enumerate}
		\item $\bigcup\limits_{r>0}$\ball{x}{r}=$X$
		\item $\bigcap\limits_{r>0}$\ball{x}{\frac{1}{r}}=$\{x\}$ \\
	\end{enumerate}
\end{lemma}

\begin{proof}
$ $\newline
\vspace{-0.15in}
	\begin{enumerate}
 		\item $\bigcup\limits_{r>0}$\ball{x}{r}=$X$\\
		\begin{enumerate}[label=(\roman*)]
			\item As \ball{x}{r} $\subset X$ for any $r$ by definition, it is obvious that $\bigcup\limits_{r>0}$\ball{x}{r} $\subset X$.
			\item Let $y \in X$, then $\exists n \in \mathbb{N}$ such that $d(x,y) < n$. Then $y \in$ \ball{x}{n} where \ball{x}{n} $\subset \bigcup\limits_{r>0}$\ball{x}{r}. Thus $y \in$ \ball{x}{n} $\subset \bigcup\limits_{r>0}$\ball{x}{r}, which implies that $\bigcup\limits_{r>0}$\ball{x}{r} $\supset X$.\\
		\end{enumerate}
		\item $\bigcap\limits_{r>0}$\ball{x}{\frac{1}{r}}=$\{x\}$ \\
		\begin{enumerate}[label=(\roman*)]
			\item As for $\forall r > 0$, $x \in$ \ball{x}{r} which makes $\{x\} \subset \bigcap\limits_{r>0}$\ball{x}{r} obvious.
			\item Suppose $x \neq y$ for $y \in \bigcap\limits_{r>0}$\ball{x}{r}. Then since $d(x,y)>0$, there exists some $n\in \mathbb{N}$ such that $0<\frac{1}{n}<d(x,y)$. Then $y \notin$\ball{x}{\frac{1}{n}}, which contradicts our assumption that $x \neq y$. Thus $\forall y \in \bigcap\limits_{r>0}$\ball{x}{r} $:y=x$, which implies that $\{x\} \supset \bigcap\limits_{r>0}$\ball{x}{r}.
		\end{enumerate}
	\end{enumerate}
\end{proof}

\begin{definition}
	Let $(X,d)$ be a metric space where $Y \subseteq X$ and $r \in \mathbb{R}$. We define, 
	\begin{enumerate}
		\item For $x \in Y$, $x$ is an \textbf{interior point} of $Y$ if $\exists r > 0$ such that \ball{x}{r} $\subset Y$.
		\item The \textbf{interior of Y}, where we denote it as $int(Y)=\{y\in Y$ $|$ $\exists r > 0 :$ \ball{y}{r} $\subset Y \}$, which is the set of all interior points of $Y$.
		\item If $int(Y)=Y$, we say that $Y$ is \textbf{open in X}, or an \textbf{open subset of X}. \\ 
	\end{enumerate}
\end{definition}

\begin{remark}
	By definition, $int(A) \subset A$.\\
\end{remark}

By defining the concept of open balls, we just introduced the concept of open subsets of a certain metric space. Let's take a look at some examples to get close with these concepts.\\ 

\begin{example}
$ $\newline
\vspace{-0.15in}
	\begin{enumerate}[label=\arabic*)]
		\item $X \subseteq X $ : $int(X)=X$
		\vskip 0.5ex
		Then by the definition of open balls it is obvious that $int(X)=X$. To elabporate, for any $x \in X$ 
		\item $Y=[0,1)\subset\mathbb{R}$ : $int(Y)=(0,1)$.
		\vskip 0.5ex
		This can be shown directly by using the definitions above. 0 is not an interior point as $\forall r>0$, \ball{0}{r}$=(-r,r)\not\subset [0,1)$. Also $\forall x \in (0,1)$, $x$ is an interior point. This is due to $\forall x, \exists r=min\{x,1-x\}$ such that \ball{x}{r} $\subset[0,1)$.
		\item $Y=[0,1) \subset X=[0,\infty)$ : $int(Y)=[0,1)$.
		\vskip 0.5ex
		Compared to the previous case, the point $0$ is in $int(Y)$. This is because as for any $r>0$, \ball{0}{r} $= \{x\in[0,\infty) : d(0,x)<r \}=[0,r)$. 
		\item $Y=\{(x,0):0<x<1\} \subset \mathbb{R}^2$ : $int(Y)=\phi$.\\
		\end{enumerate}
\end{example}

The second and third example compared shows us that \textbf{whether such subset is open or not depends on the total metric space where the subset lies in.} After taking a look of some examples above, it's natural to have questions about which are open subsets, regardless of the choice of the total metric space. With some insight and the name itself, it seems as if every open balls should be open sets. The next theorem exactly states that every open ball is open. \textit{(very surprising)}\\

\begin{theorem}
	Let $(X,d)$ be a metric space where $x \in X , r>0$. The \textbf{open ball \ball{x}{r} is open} in $X$.\\
\end{theorem}
\begin{proof}
	Let $ y \in$ \ball{x}{r}, and $ r' = r - d(x,y) > 0$. Also let $ z\in$ \ball{y}{r'}. We will show that $z \in$ \ball{x}{r} which is equivalent of showing $d(x,z)<r$. Using the triangular inequality,
	\begin{equation}\nonumber
		d(x,z) \leq d(x,y)+d(y,z)\\
	\end{equation}
	As $z \in$ \ball{y}{r'}, $d(y,z)<r'$. Using this relation we obtain, 
	\begin{equation}\nonumber
	d(x,z) \leq d(x,y)+d(y,z)<d(x,y)+r'=r\\
	\end{equation}
	As $d(x,z)<r$, it implies that $z\in$ \ball{x}{r}. Thus, as $\forall z\in$ \ball{y}{r'} $\implies z \in$ \ball{x}{r}, we can conclude that \ball{y}{r'} $\subset$ \ball{x}{r}. Again this shows us that $\forall y \in$ \ball{x}{r} $\implies y \in int(B(x;r))$. Thus $B(x;r)=int(B(x;r))$.\\
\end{proof}

\begin{example}
	Consider a metric space $(X,d)$ where $X$ is a set and $d$ is the discrete metric. Open balls with radius 1 in this metric space, $B(x,1)=\{x\}$ are all open sets in $X$. In other words, every singleton subset are open sets in $X$ when discrete metric is given.\\ 
\end{example}

As we've seen that every open balls are open, now we will show that a union of open balls is also open. This lets us consider open balls as building blocks of open sets, which is quite acceptable.\\

\begin{theorem}
	Let $(X,d)$ be a metric space where $U \subset X$. Then $U$ is open in $X$ iff $U$ is an union of open balls in $X$. \\
\end{theorem}

\begin{proof}
$ $\newline
\vspace{-0.15in}	
	\begin{enumerate}[label=(\roman*)]
		\item $U$ is an union of open balls in $X$ $\implies$ $U$ is open in $X$.
		\vskip 0.5ex
		Let $U=\bigcup\limits_{x \in A} B(x;r_x)$ for some $A \subset U$. Let $y\in U$, so $y\in B(x;r_x)$ for some $x\in A$. Since $B(x;r_x)$ is open, there exists some $r>0$ for $y$ such that $B(y;r)\subset B(x;r_x)$\\
		\item $U$ is an union of open balls in $X$ $\impliedby$ $U$ is open in $X$.
		\vskip 0.5ex
		Suppose $U$ is open in $X$, then for each $x \in U$ there exists $r_x>0$ such that $B(x;r_x)\subset U$. We claim that  $U=\bigcup\limits_{x \in U} B(x;r_x)$ which is a union of open balls. First, $U=\bigcup\limits_{x \in U} \{x\} \subset \bigcup\limits_{x \in U} B(x;r_x)$ as for every $x$, $\{x\}\subset B(x;r_x)$. The other side can be also shown as $B(x;r_x)\subset U \implies \bigcup\limits_{x \in U} B(x;r_x) \subset U$.
	\end{enumerate}
\end{proof}

Yeah and why do we only have to think of open sets? Let us also define closed sets. Because that surely makes sense. I don't know, I feel so. And you should feel so too. That's how mathematics works. \textit{(I am majoring in Physics)}\\

\begin{definition}
	Let $(X,d)$ be a metric space, where $Y \subset X$ and $r\in\mathbb{R}$. We define,
	\begin{enumerate}
		\item $x \in X$ is said to be \textbf{adherent to Y} if for $\forall r>0$, $B(x;r)\cap Y\neq\phi$.
		\item The \textbf{closure of Y} in $X$ is denoted as $\overline{Y}$, and defined as $\{x\in X$ $|$ $\forall r>0$, $B(x;r)\cap Y\neq\phi  \}$, which is the set of all points adherent to $Y$.
		\item $Y$ is said to be \textbf{closed in X} if $Y=\overline{Y}$.\\
	\end{enumerate}
\end{definition}

\begin{remark}
	Again, by definition $Y \subset \overline{Y}$ always holds. Thus $int(Y)\subset Y\subset \overline{Y}$ also always holds.
\end{remark}

Again as we've introduced some new concepts let's get used to it by taking a close look to some simple examples.\\
\begin{example}
$ $ \newline
\vspace{-0.15in}
\begin{enumerate}[label=\arabic*)]
	\item yeet
	\item yeet
	\item yeet\\
\end{enumerate}
\end{example}

\begin{theorem}
	For $x \in X$, a singleton subset $\{x\}$ is closed in $X$.\\
\end{theorem}

\begin{theorem}
	 $X$ and $\phi$ are both open and closed in $X$.\\
\end{theorem}

\begin{example}
	For a metric space where the discrete metric is given, every subset are open and closed in it. We've shown that every singleton subset is an open set as they are open balls, and \textbf{Thm.4} states that those are also all closed at the same time. Again we've also seen that a union of open sets are open, which makes every subset open as they can be expressed as a union of singleton subsets. \\
\end{example}

Open and closed sets are weird, not open doesn't imply that it is closed, and conversely not closed also doesn't imply that it is open. The next theorem kind of lets us make a connection to open and closed subsets.\\

\begin{theorem}
\textbf{Very Important!} \\
$ $ \newline 
 For a metric space $X$ where $Y \subset X$, $Y$ is closed in $X$ iff $X\setminus Y$ is open in $Y$.\\
\end{theorem}

\begin{proof}
$ $ \newline
\vspace{-0.15in}
	\begin{enumerate}[label=(\roman*)]
		\item $Y$ is closed in $X$ $\implies$ $X\setminus Y$ is open in $Y$.
		\vskip 0.5ex
		Let $x \in X \setminus Y$ which implies that $x \notin Y$. Since $Y$ is closed, $Y=\overline{Y}$, thus $x \notin \overline{Y}$. In other words, $x$ is not adjacent to $Y$, which means that there exists some $r>0$ such that $B(x;r)\cap Y = \phi$. This again implies that $B(x;r) \subset X \setminus Y$. Thus $x \in int(X \setminus Y)$, and as this holds for $\forall x \in X \setminus Y$, we can conclude that $X\setminus Y \subset int(X \setminus Y)$. Thus $X \setminus Y$ is open in $Y$.\\
		\item $Y$ is closed in $X$ $\impliedby$ $X\setminus Y$ is open in $Y$.
		\vskip 0.5ex
		Let $x \in X \setminus Y$, and as $X \setminus Y$ is open, there exists $r>0$ such that $B(x;r) \subset X \setminus Y \iff B(x;r)\cap Y = \phi \iff x \text{ is not adherent to } Y \iff x \notin \overline{Y} \iff x \in X \setminus \overline{Y}$. Thus as $x \in X \setminus Y \implies x \in X \setminus \overline{Y}$, it implies that $\overline{Y} \subset Y$. Thus $Y$ is closed in $X$.
	\end{enumerate}
\end{proof}

Of course this has a lot of mathematical significance but another great thing about this is that it can somehow ease some cumbersome "is it open or closed?" questions by taking its complement over the total metric space. For example, (actually this is not that complicated)\\

\begin{example}
	$Y=[0,1]$ is closed in $\mathbb{R}$ as $\mathbb{R} \setminus Y = (-\infty,0)\cup(1,\infty)$ is open as it is a union of open sets.\\
\end{example}

\begin{theorem}
For $Y\subset X$, where $X$ is a metric space,
\begin{enumerate}
	\item $int(Y)=int(int(Y))$ , which implies that \textbf{any interior is open}.
	\item $\overline{(\overline{Y})}=\overline{Y}$, which implies that \textbf{any closure is closed}.\\
\end{enumerate}

\begin{remark}
	This actually leads us to the fact that (of course with some additional proof), for a subset $Y $of a metric space, $int(Y)$ is the \textbf{largest open set} contained in $Y$ and $\overline{Y}$ is the \textbf{smallest closed set} containing $Y$. \\
\end{remark}
\end{theorem}

Now knowing which are open and closed, and what kind of relations they have, as we've done with the union of open balls is open, we want to see when does the open and closedness is conserved under certain set operations. The following theorem states those certain conditions.\\

\begin{theorem}
\textbf{Very Important!} 
\begin{enumerate}
	\item A union of open sets of $X$ is open.
	\item An \textbf{intersection of finitely many open sets} of $X$ is open.
	\item An intersection of closed sets of $X$ is closed.
	\item A \textbf{union of finitely many open sets} of $X$ is closed. \\
\end{enumerate}
\end{theorem}

Proving 1 and 2 suffices to show 3 and 4 using the open and closed relation. \\

\begin{proof}
$ $ \newline
\vspace{-0.15in}
\begin{enumerate}
	\item Let $U=\bigcup\limits_{\alpha \in A} U_\alpha$ where each $U_\alpha$ is open in $X$. Suppose $x \in U$, then there exists some $\alpha_0\in A$ such that $x \in U_{\alpha_0}$. Since $U_{\alpha_0}$ is open, there exists $r>0$ such that $B(x;r) \subset U_{\alpha_0} \subset U$, which implies that $B(x;r) \subset U$. As $x \in int(U)$ for every $x$ in $U$, so $U \subset int(U)$. Thus $U$ is open.
	\vskip 0.5ex
	\item Let $U=\bigcap\limits^n_{i=1} U_i$ where each $U_i$ is open in $X$. Suppose $x \in U$. Then $x \in U_i$ for each $i=1,2,\dots,n$. Since each $U_i$ is open, there exists each $r_i>0$ such that $B(x;r_i)\subset U_i$. Take $r=min\{r_1, \dots ,r_n \} $, then $B(x;r) \subset B(x;r_i) \subset U_i$. Thus $B(x;r)\subset \bigcap\limits^n_{i=1}U_i = U$ which implies that $x \in int(U)$, and as it holds for any $x\in U$, $U$ is open.
\end{enumerate}
\end{proof}

This sums up to the fact that open sets are still open after arbitrary union and closed sets are closed after arbitrary intersections, while for open sets adding a finite condition while intersecting gurantees being open and for closed sets vice versa. \\

\subsection{Completeness}
\vspace{2.5ex}

\begin{recall}
	A sequence $\{a_n\}$ in $\mathbb{R}$ converging to $a$ implies that for given $\epsilon >0$ there exists $N>0$ such that $|a-a_n|<\epsilon$ for $\forall n\geq N$.\\
\end{recall}

\begin{definition}
	Let $(X,d)$ be a metric space where $\{x_n\}^\infty_{n=0}$ is a sequence in $X$. We say that$\{x_n\}$ \textbf{converges} to $x \in X$ if for given $\epsilon>0$ there exists $N>0$ such that $d(x,x_n)<\epsilon$ for $\forall n\geq N$, i.e, $\lim\limits_{n \to \infty} d(x,x_n)=0$. In this case we denote it as $\lim\limits_{n \to \infty} x_n = x$ and say that $x$ is the \textbf{limit} of $\{x_n\}$.  \\
\end{definition}

Time to check super trivial shit. Of course lmao if some sequence has a limit it must be unique. The next theorem asserts it.\\

\begin{theorem}
	Let $(X,d)$ be a metric space and $\{x_n\}^\infty_{n=0}$ a sequence in $X$. If $\lim\limits_{n \to \infty} x_n = x$ and $\lim\limits_{n \to \infty} x_n = y$, then $x=y$. In other words, if a sequence converges, its \textbf{limit uniquely exists.}\\
\end{theorem}

\begin{proof}
	Suppose that $x \neq y$. Let $r=\frac{1}{2} d(x,y) >0$. Since $x_n \to x$, $\exists N>0$ such that $x_n \in B(x;r)$ for $\forall n \geq N$. Also, at the same time as $x_n \to y$, $\exists M>0$ such that $x_n \in B(y,r)$ for $\forall n \geq M$. Then for $\forall n \geq max\{N,M\}$, $x_n \in B(x;r)\cap B(y,r)$. But $B(x;r)\cap B(y,r)=\phi$ as if $\exists z \in B(x;r)\cap B(y,r)$ ,
	\begin{equation} \nonumber
		d(x,y) \leq d(z,x) + d(y,z) < 2r = d(x,y)
	\end{equation} 
	which contradicts $\exists z \in B(x;r)\cap B(y,r)$, thus $x_n \in B(x;r)\cap B(y,r)=\phi$ which again contradicts the very first assumption we made. Thus $x=y$.\\
\end{proof}

\begin{theorem}
	$x$ is adherent to $Y \iff \exists \{y_n\}^\infty_{n=0}$ in $Y$ such that $y_n \to x$.\\
\end{theorem}

\begin{proof}
$ $ \newline
\vspace{-0.15in}
\begin{enumerate}[label=(\roman*)]
	\item $x$ is adherent to $Y \implies \exists \{y_n\}^\infty_{n=0}$ in $Y$ such that $y_n \to x$
	\vskip 0.5ex
	Since $x$ is adherent to $Y$, for each $n\in \mathbb{N}$, $B(x;\frac{1}{n}) \cap Y \neq \phi$. This implies that there exists $y_n \in Y$ such that also $y_n \in B(x;\frac{1}{n})$, i.e, $\exists y_n \in B(x;\frac{1}{n}) \cap Y$. We now claim that for this sequence, $y_n \to x$. Since $y_n \in B(x;\frac{1}{n})$, 
	\begin{equation}\nonumber
		d(y_n,x)<\frac{1}{n} \implies 0 \leq \lim\limits_{n \to \infty} d(x_n,y) \leq \lim\limits_{n \to \infty} \frac{1}{n} =0
	\end{equation}
	Thus as $d(y_n,x) \to 0$, $y_n \to x$ by definition.
	\item $x$ is adherent to $Y \impliedby \exists \{y_n\}^\infty_{n=0}$ in $Y$ such that $y_n \to x$
	\vskip 0.5ex
	Since $y_n \to x$, there exists $N>0$ such that $d(y_n,x)<r$ for $\forall n\geq N$ and some $r>0$. In particular, as $y_N \in Y$ and $d(y_N,x)<r$, i.e, $y_N \in B(x;r) \cap Y$. Thus as $B(x;r) \cap Y \neq \phi$, $x$ is adherent to $Y$.
\end{enumerate}
\end{proof}

\begin{definition}
	Let $X$ be a metric space equipped with a metric $d$, and $Y \subset X$.
	\begin{enumerate}
		\item $x \in X$ is a \textbf{limit point} of $Y$ if $B(x;r) \cap Y$ is infinite for $\forall r>0 \iff \left(B(x;r)\setminus \{x\}\right)\cap Y \neq \phi$ for $\forall r>0$.
		\item $x \in Y$ is an \textbf{isolated point} of $Y$ if $\exists r>0$ such that $B(x;r)\cap Y = \{x\}$ $\iff \left(B(x;r)\setminus \{x\}\right)\cap Y = \phi$ for $\exists r>0$.\\
	\end{enumerate}
\end{definition}

\begin{remark}
	$x$ is a limit point of $Y$ $\iff$ $x$ is a limit of a sequence in $Y$.\\
\end{remark}

As you can see on \textbf{Def.7}, being a limit point is a stronger condition than being adherent to the same subset.\\

\begin{example}
$ $ \newline
\vspace{-0.15in}
	\begin{enumerate}[label=\arabic*)]
		\item $Y=(0,1] \cup \{2\}$ in $\mathbb{R}$ with the standard metric.
		\begin{itemize}
			\item limit points : $[0,1]$
			\item isolated points : $\{2\}$
		\end{itemize}
		\vskip 0.5ex
		\item  Same $Y$ in $\mathbb{R}$ but with the discrete metric.
		\begin{itemize}
			\item limit points : $\phi$
			\item isolated points : $Y$\\
		\end{itemize}
	\end{enumerate}
\end{example}


\begin{theorem}
	For $Y \subset X$ where $X$ is a metric space, $\overline{Y}= \mathcal{L} \bigsqcup \mathcal{I}$ where $\mathcal{L}$ is the set of limit points of $Y$ and $\mathcal{I}$ is the set of isolated points in $Y$.\\
\end{theorem}

\begin{remark}
	By definition, $\mathcal{L}\cap\mathcal{I}=\phi$. In other words, a point can't be a limited point while being an isolated point at the same time.\\
\end{remark}

\begin{definition}
	Let $Y$ be a subset of a metric space $X$. The \textbf{boundary} of $Y$, denoted as $\partial Y$ is defined as $\partial Y = \overline{Y} \cap \overline{X \setminus Y}$.\\
\end{definition}

\begin{theorem}
	Let $X$ metric space, where $U \subset Y \subset X$. Then,\\
	\begin{equation} \nonumber
		U \text{ is open in } Y \iff \text{there exists an open set } V \text{ of }X \text{ s.t. } U = Y \cap V \\
	\end{equation}
\end{theorem}

\begin{proof}
$ $ \newline
\vspace{-0.15in}
\begin{enumerate}[label=(\roman*)]
	\item $(\implies)$
	\vskip 0.5ex
	For each $x \in U$ there exists $r_x>0$ such that $B(x;r_x)\cap Y \subset U$ since $U$ is open in $Y$. Let $V = \bigcup_{x\in U} B(x;r_x)$. Since such $V$ is an union of open balls of $X$, $V$ is open in $X$. Now we claim that $V \cap Y = U$.
	\vskip 0.5ex
	\begin{enumerate}
		\item $ U \subset V \cap Y $ \vskip 0.5ex
		By assumption, $U \subset Y$ holds and as
		\begin{equation}\nonumber
			U = \bigcup\limits_{x \in  U} \{x\} \subset  \bigcup\limits_{x \in  U} B(x;r_x) = V
		\end{equation}
		$U \subset V$ also holds. Thus $U \subset V \cap Y$.\\
		
		\item $V \cap Y \subset U$ 
		\begin{equation}\nonumber
			\begin{split}
			V \cap Y &= \left( \bigcup\limits_{x \in U}  B(x;r_x)\right)\cap Y \\
			&=  \bigcup\limits_{x \in U} \left(  B(x;r_x) \cap Y \right) \subseteq U.
			\end{split}
		\end{equation}
	\end{enumerate} 
	\item  $(\impliedby)$
	\vskip 0.5ex
	Let $x\in U$. Since $Y \cap V = U$, it also implies that $x \in V$. Since $V$ is open in $X$, there exists $r_x > 0 $ for each $x\in V$ that $B(x;r_x) \subset V$. Then $B(x;r_x) \cap Y \subset V \cap Y = U $. Thus as $\forall x \in U$, $\exists r_x > 0$ such that $B(x;r_x) \cap Y \subset U$, $U$ is open in $Y$.
\end{enumerate}
\end{proof}

\begin{definition}
Let $(X,d)$ be a metric space where $\{x_n\}$ is a sequence in $X$.
\begin{enumerate}
	\item A sequence $\{x_n\}$ is a \textbf{Cauchy sequence} if $\lim\limits_{n,m \to \infty} d(x_n,x_m) = 0$. \vskip 0.5ex
	$\iff$ for a given $\epsilon > 0 $ there exists $N > 0$ s.t. $d(x_n,x_m)<\epsilon$ for $\forall n,m \geq N$.
	\item $X$ is \textbf{complete} if every Cauchy sequence converges in $X$.\\
\end{enumerate}
\end{definition}

\begin{remark}
	$\mathbb{R}$ is complete, it is one of the well known properties of  $\mathbb{R}$.
\end{remark}

\begin{example}
	Consider $A=(0,\infty)$ and $x_n = \frac{1}{n}$ as a sequence. Then $\{x_n\}$ is a Cauchy sequence in $A$. For a given $\epsilon >0$, there exists $N>0$ such that $0<\frac{1}{n}<\epsilon$ for any $n\geq N$. Then $d(x_n, x_m) = \big|\frac{1}{n}-\frac{1}{m}\big| < \epsilon$ for all $n,m \geq N$. However, while $\{x_n\}$ converges in $\mathbb{R}$, it doesn't converge in $A$ as $x_n \to 0 \notin A$. Thus we can conclude that $A$ is not complete, as there exists a Cauchy sequence in $A$ that does not converge in $A$.\\
\end{example}

\begin{theorem}
Let $X$ be a metric space.
\begin{enumerate}
	\item A convergent sequence in $X$ is a Cauchy sequence.
	\item If a Cauchy sequence in $X$ has a convergent subsequence, then the sequence converges in $X$.\\
\end{enumerate}	
\end{theorem}

\begin{proof}
$ $ \newline
\vspace{-0.15in}
\begin{enumerate}
	\item Let $\{x_n\}$ be a sequence in $X$ where $x_n \to x$. Suppose $\epsilon >0$ is given. Since $x_n \to x$, there exists some $N>0$ such that $d(x,x_n)<\epsilon/2$ for all $n \geq N$. Then,
	\begin{equation}\nonumber
		d(x_n.x_m) \leq d(x_n,x) + d(x,x_m) < \epsilon/2 + \epsilon/2 = \epsilon
	\end{equation}
	for any $n,m \geq N$, which makes $\{x_n\}$ a Cauchy sequence.
	\item Let $\{x_{n_k}\}$ be the convergent subsequence of $\{x_n\}$, and let the limit as $x_{n_k} \to x$. Since $\{x_n\}$ is a Cauchy sequence, for a given $\epsilon/2 >0$ there exists $N_1>0$ such that $d(x_n,x_m)<\epsilon/2$ for all $n,m \geq N$. Also as $\{x_{n_k}\}$ converges, there also exists some $N_2>0$ such that $d(x,x_{n_k})<\epsilon/2$ for all $n_k \geq N_2$. Take $N=max\{N_1,N_2\}$. Then for $n\geq N$,
	\begin{equation}\nonumber
		d(x_n,x) \leq d(x_n,x_{n_k}) + d(x_{n_k},x) < \epsilon/2 + \epsilon/2 = \epsilon
	\end{equation}
	Thus $\{x_n\}$ converges in $X$.
\end{enumerate}
\end{proof}

\begin{remark}
	Note that being complete is independent of where it lies on, while being closed depends on its outer space.\\
\end{remark}

The next theorem lets us think about the relation between being closed and complete.\\

\begin{theorem} Let $X$ be a metric space.
$ $ \newline
\vspace{-0.15in}
	\begin{enumerate}
		\item If $X$ is complete and $Y$ is a closed subset of $X$, then $Y$ is also complete.
		\item If $Y$ is a complete subset of $X$ then it is closed.\\
	\end{enumerate}
\end{theorem}

\begin{proof}
$ $ \newline
\vspace{-0.15in}
\begin{enumerate}
	\item Let $\{y_n\}$ be a Cauchy sequence in $Y$. Then it is also a Cauchy sequence in $X$. Since $X$ is complete, $\{y_n\}$ converges in $X$. To elaborate, there exists $x \in X$ such that $y_n \to x$. Since $Y$ is closed in $X$ and $x$ is a limit of a sequence in $Y$, $x \in Y$. Thus as there exists a limit $x\in Y$ for every Cauchy sequences in $Y$, $Y$ is also complete. 
	\item Suppose $\{y_n\}$ is a sequence in $Y$ which converges in $X$ such that $y_n \to x \in X$. Since $\{y_n\}$ is a convergent sequence in $X$, it is also a Cauchy sequence in $X$ due to \textbf{Thm 13}, hence it is also a Cauchy sequence in $Y$. Since $Y$ is complete, $\{y_n\}$ converges in $Y$, i.e, $\exists y\in Y$ such that $y_n \to y$. Due to \textbf{Thm 9}, the limit of $\{y_n\}$ uniquely exists, thus $x=y\in Y$. Thus $Y$ is closed.
\end{enumerate}
\end{proof}

\begin{definition}
	A subset $A$ of a metric space $X$ is \textbf{dense} in $X$ if $\overbar{A}=X$.\\
\end{definition}

\begin{remark}
	The above definition is also equivalent to, 
	\begin{enumerate}
		\item $U \cap Y \neq \phi$ for every open set $U$ of $X$.
		\item $B(x;r) \cap Y \neq \phi$ for every $x \in X, r>0$.\\
	\end{enumerate}
\end{remark}

For the first statement of the above remark, we state a proof below.\\

\begin{proof}
$ $ \newline
\vspace{-0.15in}
	\begin{enumerate}[label=(\roman*)]
		\item $(\implies)$ 
		\vskip 0.5ex
		Let $x \in X$ and $r>0$. Since $\overbar{A}=X$, $x \in \overbar{A}$, which implies that $x$ is adherent to $A$. So by definition of adherent points, for any $r>0$ $B(x;r)\cap Y \neq \phi$
		\item  $(\impliedby)$
		\vskip 0.5ex
		Let $x \in X$. By our assumption, it implies that $x$ is adherent to $A$. Thus $x \in \overbar{A}$
	\end{enumerate}	
\end{proof}

\begin{definition}
	For $A \subset X$, $A$ is \textbf{nowhere dense} in $X$ if $int(\overbar{A})=\phi$.
\end{definition}

\vspace{0.15in}
\subsection{Compactness}
\vspace{0.15in}

\begin{definition}
	Let $X$ be a metric space.
	\begin{enumerate}[label=\arabic*)]
		\item Let $U_\alpha$ be subsets of $X$. Then $\{ U_\alpha \}_{\alpha \in A}$ is a \textbf{cover} of $X$ if $\bigcup\limits_{\alpha \in A} U_\alpha = X$.
		\item A cover $\{ U_\alpha \}_{\alpha \in A}$ is an \textbf{open cover} if all $U_\alpha$ are open in $X$.
		\item Let $\{U_\alpha \}_{\alpha \in A}$ be a cover of $X$. Then $\{ V_\beta \}_{\beta \in B}$ is a \textbf{subcover} of $\{U_\alpha \}_{\alpha \in A}$ if $\{ V_\beta \}_{\beta \in B} \subset \{U_\alpha \}_{\alpha \in A}$ and $\bigcup\limits_{\beta \in B} V_\beta = X$.
		\item Let $\{U_\alpha \}_{\alpha \in A}$ be a cover of $X$. Then $\{ V_\beta \}_{\beta \in B}$ is a \textbf{finite subcover} of $\{U_\alpha \}_{\alpha \in A}$ if $\{ V_\beta \}_{\beta \in B}$ is a subcover of $\{U_\alpha \}_{\alpha \in A}$ and $B$ is finite. \\
	\end{enumerate}
\end{definition}

\begin{example}
	content...\\
\end{example}

\begin{definition}
	A metric space $X$ is \textbf{compact} if every open cover has a finite subcover.\\
\end{definition}

\begin{remark}
	Let $X$ be a metric space and $Y$ be a subspace of $X$.\\
\end{remark}

\begin{theorem}
	The following are equivalent.
	\begin{enumerate}[label=\arabic*)]
		\item $X$ is compact. (cover sense)
		\item Every sequence in $X$ has a convergent subsequent. 
		\item $X$ is complete and totally bounded.\\
	\end{enumerate}
\end{theorem}

\begin{proof}
$ $ \newline
\vspace{-0.15in}
\begin{enumerate}[label=(\arabic*)]
	\item  1) $\implies$ 2)
	\vskip 0.5ex
	Let $\{x_n\}$ be a sequence in $X$ such that any subsequence of $\{x_n\}$ doesn't converge. Then for each $x\in X$ there exists $\epsilon_x >0$ such that $B(x,\epsilon_x)$ contains only finitely many terms. (if not, there exists a converging subsequence). Then $X = \bigcup\limits_{x \in X} B(x,\epsilon_x)$, which implies that $\{B(x,\epsilon_x)\}_{x\in X}$ is an open cover of $X$. As $X$ is compact by assumption, there exists a finite subcover of  $\{B(x,\epsilon_x)\}_{x\in X}$. Let such finite subcover $\{B(y_i,\epsilon_{y_i})\}_{i=1}^n$ by some $y_1, \dots , y_n \in X$ such that $X = \bigcup\limits_{i=1}^n B(y_i,\epsilon_{y_i})$ for some finite $n \in \mathbb{N}$. Since $\{x_n\}_{n=1}^\infty \subset X$ we have $\{x_n\}_{n=1}^\infty \subset \bigcup\limits_{i=1}^n B(y_i,\epsilon_{y_i})$ which induces a contradiction as $B(x,\epsilon_x)$ contains only finitely many terms and a finite union of such $B(y_i,\epsilon_{y_i})$ couldn't contain such series as a subset as the series has infinite terms.
	\item 2) $\implies$ 3) (only completeness)
	\vskip 0.5ex
	Let $\{x_n\}$ be a Cauchy sequence in $X$. By assumption $\{x_n\}$ has a subsequence $\{x_{n_k}\}$ which converges in $X$. As a Cauchy sequence $\{x_n\}$ has a convergent subseqeunce, $\{x_n\}$ is a converging Cauchy sequence. Thus as any Cauchy sequence converges in $X$, $X$ is complete.
\end{enumerate}
\end{proof}

\begin{remark}
	compactness $\implies$ completeness $\implies$ closedness \\
\end{remark}

\begin{theorem}
	In $\mathbb{R}^n$ let $X$ be a subspace of it. Then TFAE, 
	\begin{enumerate}[label=\arabic*)]
		\item $X$ is compact.
		\item Every sequence in $X$ has a converging subsequence.
		\item $X$ is closed and bounded. \\
	\end{enumerate}
\end{theorem}

The above theorem is also known as the well known \textbf{Heine-Borel theorem.}\\

\begin{lemma}
	Let $X \subset \mathbb{R}^n$.
	\begin{enumerate}[label=\arabic*)]
		\item $X$ is complete $\iff$ $X$ is closed in $\mathbb{R}^n$.
		\item $X$ is totally bounded $\iff$ $X$ is bounded. \\
	\end{enumerate}
\end{lemma}

For general metric spaces, the above doesn't hold, but rather the left to right implication holds.\\

\begin{definition}
	Let $X$ be a metric space. 
	\begin{enumerate}[label=\arabic*)]
		\item $X$ is \textbf{bounded} if there exists $r>0$ such that $d(x,y)<r$ for all $x,y \in X$.
		\item $X$ is \textbf{totally bounded} if for each $\epsilon > 0$ there are a finite number of open balls $B(x_1 , \epsilon)$, $B(x_2, \epsilon)$, $\dots$ , $B(x_n, \epsilon)$ such that \\
		\begin{equation}\nonumber
			X = \bigcup\limits_{i=1}^n B(x_i,\epsilon). \\
		\end{equation}
	\end{enumerate}
\end{definition}

\begin{lemma}
	For $Y\subset X$, 
	\begin{enumerate}[label = \arabic*) ]
		\item $X$ is bounded $\implies$ $Y$ is bounded. 
		\item $X$ is totally bounded $\implies$ $Y$ is totally bounded.\\
	\end{enumerate}
\end{lemma}

\begin{example}
	content...\\
\end{example}

\vspace{0.15in}
\subsection{Seperable, Second Countable, Basis}
\vspace{0.15in}

\begin{definition}
	A metric space $X$ is \textbf{seperable} if $X$ has a countable dense subset.
\end{definition}

\vspace{0.15in}

\begin{example}
\nextline
\begin{enumerate}
	\item A countable metric space is seperable. (trivial)
	\item $\mathbb{R}$ is seperable as $\exists \mathbb{Q}\subset \mathbb{R}$ such that $\overbar{\mathbb{Q}}=\mathbb{R}$.
	\item $\mathbb{R}^n$ is also seperable as $\exists \mathbb{Q}^n \subset \mathbb{R}^n$ such that $\overbar{\mathbb{Q}^n}=\mathbb{R}$.
	\item A compact metric space is seperable.
	\item If $X$ is seperable and $A \subset X$ then $A$ is also seperable.
\end{enumerate}
\end{example}

\vspace{0.15in}

\begin{definition}
	Let $X$ be a metric space. Let $\mathcal{B}$ be a family of some open subsets of $X$. Then $\mathcal{B}$ is a \textbf{base} if every open subset of $X$ is a union of subsets in $\mathcal{B}$. 
\end{definition}

\vspace{0.15in}

\begin{example}
\nextline
\begin{enumerate}
	\item The family of all open balls of $X$ is a base of $X$.
	\item The family of all open sets of $X$ is a base of $X$. (trivial)\\
\end{enumerate}
\end{example}

\begin{lemma}
	Let $\mathcal{B} \subset \mathcal{P}(X)$ where $X$ is a metric space. Then $\mathcal{B}$ is a base for open sets of $X$ iff for each $x \in X$ and each open neighborhood $U$ of $x$, there exists $V_x \in\mathcal{B}$ such that $x \in V_x \subset U$.
\end{lemma}

\begin{proof}
\nextline
\begin{enumerate}[label=\roman*)]
	\item $(\implies)$
	\vskip 0.5ex
	Let $U$ be an open neighborhood of $x\in X$. As $\mathcal{B}$ is a base, $U=\bigcup\limits_{\alpha \in A}V_\alpha$ for some $V_\alpha \in \mathcal{B}$. Then for some $\alpha'\in A$, as $x \in U= \bigcup\limits_{\alpha \in A}V_\alpha \implies x \in V_{\alpha'} \subset U$ where $V_{\alpha'}\in \mathcal{B}$.
	\item $(\impliedby)$
	\vskip 0.5ex
	Let $U$ be an open subset of $X$. For each $x\in U$ there exists $V_x \in \mathcal{B}$ such that $x \in V_x \subset U$ by assumption. Now we claim that $U = \bigcup\limits_{x\in U} V_x$ : 
	\begin{enumerate}[label=(\roman*)]
		\item $\bigcup V_x \subseteq U$ is obvious since for each $x$, $V_x \subset U$
		\item $U = \bigcup\limits_{x \in  U}\{x\} \subset \bigcup\limits_{x \in  U} V_x$
	\end{enumerate}
	which shows that our claim holds. Thus $\mathcal{B}$ is a basis.
\end{enumerate}
\end{proof}

\vspace{0.15in}

\begin{example}
Then for $\mathbb{R}$ the following are all bases :
\begin{enumerate}
	\item $\mathcal{B} = \{ \text{all open subsets of } \mathbb{R} \}$ (trivial)
	\item $\mathcal{B} = \{ B(x,r) : x\in\mathbb{R} ,r>0 \}$
	\item $\mathcal{B} = \{ B(x,r) : x\in\mathbb{R},r\in\mathbb{Q},r>0 \}$
	\vskip 0.5ex 
	Such family being a basis can be shown using \textbf{Lem 4.} : \\
	 Choose any $x\in U \subset \mathbb{R}$ where $U$ is an open neighborhood of $x$. As $U$ is open, there exists $r>0$ such that $B(x,r)\in U$. Then we can choose some $r'>0$ in $\mathbb{Q}$ such that $0<r'<r$. Then $x \in B(x,r') \subset B(x,r) \subset U$ where $B(x,r') \in \mathcal{B}$.
	 \item $\mathcal{B} = \{ B(x,r) : x,r\in\mathbb{Q},r>0 \}$
	 \vskip 0.5ex 
	 Suppose for any $x\in U \subset \mathbb{R}$ where $U$ is an open set. Then there exists $r>0$ such that $B(x,r) \subset U$
\end{enumerate}
\end{example}

\vspace{0.15in}

\begin{definition}
	A metric space is \textbf{second countable} if $X$ has a countable base.
\end{definition}

\vspace{0.15in}

\begin{theorem}
	For a metric space $X$, $X$ is seperable $\iff$ $X$ is second countable.
\end{theorem}

\vspace{0.15in}

\begin{remark}
	For a metric space $X$, $X$ is compact $\iff$ seperable $\iff$ second countable
\end{remark}

\vspace{0.15in}
\subsection{Continuous Functions}
\vspace{0.15in}

\begin{recall}
	If $f:\mathbb{R} \to \mathbb{R}$ is continuous at $x\in\mathbb{R}$, TFAE : 
	\begin{enumerate}[label=\arabic*)]
		\item For a given $\epsilon>0$ there exists $\delta > 0$ s.t $|x-y|<\epsilon \implies |f(x)-f(y)|<\delta$
		\item For a given $\epsilon>0$ there exists $\delta > 0$ s.t if $y \in B(x,\delta) \implies f(y) \in B(f(x),\epsilon)$
		\item For a given $\epsilon>0$  $\exists \delta > 0$ s.t $f(B(x,\delta))\subset B(f(x),\epsilon)$
		\item For a given $\epsilon>0$ $\exists \delta > 0$ s.t $B(x,\delta)\subset f^{-1}(B(f(x),\epsilon))$
	\end{enumerate}
\end{recall}

\vspace{0.15in}

\begin{definition}
	Let $(X,d_X)$ and $(Y,d_Y)$ be metric spaces. Also let $f:X \to Y$ be a function then :
	\begin{enumerate}[label=\arabic*)]
		\item We say that for some $x\in X$, $f$ is \textbf{continuous at} $x$ if for a given $\epsilon > 0 $ there exists $\delta>0$ such that 
		\begin{equation}\nonumber
			d_X(x,x') < \delta \implies d_Y(f(x),f(x')) < \epsilon
		\end{equation}
		which also can be expressed as, 
		\begin{equation}\nonumber
			\exists \delta>0 : B_X(x,\delta) \subset f^{-1}(B_Y(f(x),\epsilon))
		\end{equation}
		\item $f$ is said to be continuous if $f$ is continous at $x$ for every $x \in X$.
	\end{enumerate}
\end{definition}

\vspace{0.15in}

\begin{remark}
	$f:(X,d_x) \to (Y,d_Y)$ is an \textbf{isometry} if for $\forall x,y \in X$ 
	\begin{equation}\nonumber
		d_X(x,y) = d_Y(f(x),f(y))
	\end{equation}
	holds. This implies that any isometry is a continuous function but not vice versa.
\end{remark}

\vspace{0.15in}

\begin{theorem}
Let $(X,d_X)$ and $(Y,d_Y)$ be metric spaces and $f:X \to Y$ be a function. Then TFAE :
\begin{enumerate}[label=\arabic*)]
	\item $f$ is a continuous function.
	\item For each $x\in X$, if $\{x_n\}$ is a sequence in $X$ s.t $\lim\limits_{n\to \infty} x_n =x $ then $\lim\limits_{n \to \infty}f(x_n)=f(x)$.
	\item For each open subset $U$ of $Y$, the preimage $f^{-1}(U)$ is open in $X$.
	\item For each closed subset $E$ of $Y$, the preimage $f^{-1}(E)$ is closed in $X$.\\
\end{enumerate}
\end{theorem}

\begin{proof}
	We will only prove that 1) $\iff$ 3).
	\begin{enumerate}[label=\roman*)]
		\item 1) $\implies$ 3)
		\vskip 0.5ex
		Let $U$ be an open subset of $Y$. Let $x\in f^{-1}(U)$. Then there exists $\epsilon>0$ such that $f(x) \in B(f(x),\epsilon)\subset U$ as by assumption $f$ is a continuous function. It implies that 
		\begin{equation}\nonumber
			x \in f^{-1} (B(f(x),\epsilon)) \subset f^{-1} (U)	\\	\end{equation}
		Again as $f$ is continuous by assumption, there exists $\delta >0$ such that $B(x,\delta) \subset f^{-1} (B(f(x),\epsilon))$. Thus as there exists $\delta>0$ for $x\in f^{-1}(U)$ such that $B(x,\delta) \subset f^{-1}(U)$, $f^{-1}(U)$ is open. 
		\item 3) $\implies$ 1)
		\vskip 0.5ex
		Let $x \in X$ and suppose $\epsilon >0$ is given. Then as $B(f(x),\epsilon)$ is open in $Y$, by assumption $f^{-1}(B(f(x),\epsilon))$ is also open. As $x$ is in $f^{-1}(B(f(x),\epsilon))$ which is open, there exists $\delta>0$ such that $B(x,\delta)\subset f^{-1}(B(f(x),\epsilon))$. Thus $f$ is a continuous function.
	\end{enumerate}
\end{proof}

\vspace{0.15in}

\begin{theorem}
Let $X,Y$ both be metric spaces with a continuous function $f:X\to Y$. If $X$ is compact then $f(X)$ is also compact.\\
\end{theorem}

\begin{proof}
\nextline
\begin{enumerate}[label=\arabic*)]
	\item Compactness as sequential compactness :
	\vskip 0.5ex
	Let $\{y_n\}_{n=1}^\infty$ be a sequence in $f(X)$. Then there is a sequence $\{x_n\}_{n=1}^\infty$ in $X$ such that for every $i\in \mathbb{N} : f(x_i)=y_i$. Since $\{x_n\}$ is a sequence in $X$ and by assumption as it is compact, there exists a converging subsequence $\{ x_{n_k} \}$ in $X$. Let it converges to $x$, i.e, $x_{n_k}\to x$ as $k \to \infty $. Then $\{ f(x_{n_k}) \}$ is a subsequence of $\{y_n\}=\{f(x_n)\}$, which also converges. This is as $f$ is a continuous function : 
	\begin{equation}\nonumber
		\lim\limits_{k\to\infty} f(x_{n_k}) = f \left( \lim\limits_{k\to\infty} x_{n_k} \right) = f(x)
	\end{equation}
	which implies that for any sequence in $f(X)$, it has a convergent subsequence, thus $f(X)$ is compact.
	\item Compactness as cover sense :
	\vskip 0.5ex
	Let $\{U_\alpha\}_{\alpha\in A}$ be an open cover of $f(X)$ where for every $\alpha\in A$ it is open in $Y$. Then by definition, 
	\begin{equation}\nonumber
		f(X) \subset \bigcup\limits_{\alpha \in A} U_\alpha \implies X \subset f^{-1}\left( \bigcup\limits_{\alpha \in A} U_\alpha  \right) = \bigcup\limits_{\alpha \in A} f^{-1} \left(U_\alpha \right)
	\end{equation}
	Since $f$ is continuous, for each $\alpha \in A$, $f^{-1}(U_\alpha)$ are open, and as the above holds $\{f^{-1}(U_\alpha)\}_{\alpha\in A}$ suffices to be an open cover of $X$. Since $X$ is compact by assumption, there exists $\alpha_1 , \dots , \alpha_N$ such that $X \subset \bigcup\limits_{i=1}^N f^{-1}(U_{\alpha_i})$. This implies that 
	\begin{equation}\nonumber
	\begin{split}
	f(X) &\subset f \left( \bigcup\limits_{i=1}^N f^{-1}(U_{\alpha_i}) \right) \\
	&= \bigcup\limits_{i=1}^N f\left( f^{-1}\left( U_{\alpha_i} \right) \right) \\
	&\subset \bigcup\limits_{i=1}^N U_{\alpha_i}
	\end{split}
	\end{equation}
	Thus as for any open cover of $f(X)$ as there exists a finite subcover that covers $f(X)$, we can conclude that $f(X)$ is compact too.
\end{enumerate}
\end{proof}

\vspace{0.15in}
\section{Topological Spaces}
\vspace{0.15in}

\subsection{Topological Spaces}
\vspace{0.1in}

\begin{definition}
	Let $X$ be a set. Also let $\mathcal{T}\subset \mathcal{P}(X)$. Then we say $\mathcal{T}$ is a topology of $X$ if 
	\begin{enumerate}\nonumber
		\item $\phi,X \in \T$
		\item For $U_\alpha \in \T$ where $\alpha \in A$ then $\bigcup\limits_{\alpha \in A} U_\alpha \in \T$
		\item For $U_1,\dots,U_n \in \T$ then $\bigcap\limits_{i=1}^n U_i \in \T$ \\
	\end{enumerate} 
\end{definition}

The definition for a topology is simple, it is just a collection of subsets with some conditions, it should be still in it under arbitrary union but under finite intersection.

\vspace{0.15in}

\begin{definition}
	Let $\T$ be a topology for a set $X$. Then we define :
	\begin{enumerate}[label=\arabic*)]
		\item $(X,\T)$ is a topological space equipped with $\T$.
		\item Let $U \in X$. $U$ is open in $X$ if $U\in \T$.
		\item $U$ is closed in $X$ if $X \setminus U$ is open , i.e $X\setminus U \in \T$.\\
	\end{enumerate}
\end{definition}

Two topological spaces are said to be equal if they have thhe same set and topology. For instance $(\mathbb{R} , \{ \phi, \mathbb{R} \})$ and $(\mathbb{R}, \mathcal{P}(\mathbb{R}))$ are different topological spaces.\\

\begin{example}
\nextline
\begin{enumerate}[label=\arabic*)]
	\item For any set $X$, $\T = \{ \phi, X \}$ is a topology. We call such topology the \textbf{trivial topplogy} or the \textbf{indiscrete topology}.
	\item We can also let $\T = \mathcal{P}(X)$, which also becomes a topology for $X$. We call such $\T$ the \textbf{discrete topology}.\\
\end{enumerate}
\end{example}

If open sets where still open sets after arbitraty union and finite intersection, closed sets are still closed after finite union and arbitrary intersection. It implies that as for any topology the null set and the total set is in it, they are always both closed and open.\\

\begin{remark}
	Let $(X,\T)$ be a topological space.
	\begin{enumerate}
		\item $\phi,X$ are closed and open.
		\item For $X\setminus E_\alpha \in \T$ where $\alpha\in A$ i.e each $E_\alpha$ are closed ,  $\bigcap\limits_{\alpha \in A} E_\alpha$ is also closed.
		\item For $X\setminus E_1 , \dots , X\setminus E_n \in \T$, i.e each $E_i$ are closed, $\bigcup\limits_{i=1}^n E_i$ is also closed.
	\end{enumerate}
\end{remark}

\vspace{0.15in}

\begin{example}
For the following examples the topology are defined for $\mathbb{R}$.
\begin{enumerate}[label=\arabic*)]
	\item $\T=\{ (a,b) : a<b \}$ is not a topology :
	\vskip 0.5ex As unions of intervals is not necessarily an interval.
	\item Let $\T_U$ be the collection of all possible unions of $(a.b)$ where $a<b$.
	\vskip 0.5ex This suffices to be a topology for $\mathbb{R}$. We call such topology the \textbf{usual topology} or the \textbf{standard topology} for $\mathbb{R}$.
	\item Let $\T_d$ be a collection which includes all of the open sets of $(\mathbb{R},d)$, where $d$ is the usual metric. Then we can easily see that $\T_U = \T_d $. For such situation we say that $\T_U$ is \textbf{metrizable}.
	\item Now consider the trivial topology $\T_{tr} = \{ \phi , \mathbb{R} \}$.
	\vskip 0.5ex Then $\T_{tr}$ is not metrizable, i.e there doesn't exist a metric $d$ on $\mathbb{R}$ such that the collection of open sets induced by $d$ becomes $\T_{tr}$. This is because if we suppose such $d$ exists and the collection of all open subsets 
\end{enumerate}
\end{example}

\vspace{0.15in}

\begin{example}
Let $X=\mathbb{R}^n$ with $d : \mathbb{R}^n \times \mathbb{R}^n \to \mathbb{R}$ which is the usual metric. If we let $\T_d$ to be the collection of all open sets induced by $d$, it becomes a topology for $\mathbb{R}^n$, thus $(\mathbb{R}^n,\T_d)$ is a topological space.
\end{example}

\vspace{0.15in}

\begin{example}\textbf{(Cofinite Topology)} \\[0.1in]
	For a set $X$, let us define $\T = \{ U \subset X : |X\setminus U| < \infty \} \cup \{ \phi \}$. Such $\T$ becomes a topology and we call it the \textbf{cofinite topology} or the \textbf{finite conmpliment topology} for $X$. For example for $\mathbb{R}$ if $U,V$ are open under such topology it implies that :
	\begin{equation}\nonumber
	\begin{split}
		\exists N : \mathbb{R} \setminus U = \{ p_1, \dots, p_N \} &\iff U = \mathbb{R} \setminus \{ p_1, \dots, p_N \} \in \T \\
		\exists M : \mathbb{R} \setminus V = \{ q_1 , \dots , q_M \} &\iff V=\mathbb{R} \setminus \{ q_1 , \dots , q_M \} \in \T\\
	\end{split}
	\end{equation}
	Consider the union and intersection of $U,V$ : 
	\begin{equation}\nonumber
	\begin{split}
		U \cup V &= \mathbb{R} \setminus \left( \{p_i\}_{i=1}^N \cap \{q_i\}_{i=1}^M \right) \\
		U \cap V &= \mathbb{R} \setminus \left( \{p_i\}_{i=1}^N \cup \{q_i\}_{i=1}^M \right)
	\end{split}
	\end{equation}
	where the finiteness of the compliment of the union is guranteed under arbitrary union, while for the intersection it is only guranteed under finite intersections.
\end{example}

\vspace{0.15in}

\begin{definition}
	Let $(X,\T)$ be a topological space, where $x\in S \subset X$.
	\begin{enumerate}[label=\arabic*)]
		\item $S$ is an \textbf{open neighborhood} of $x$ if $S$ is open.
		\item $S$ is a \textbf{neighborhood} of $x$ if $S$ contains an open neoghborhood of $x$, i.e there exists some $U \in \T$ such that $x \in U \subset S$.
		\item $x$ is an \textbf{interior point} of $S$ if there exists an open set $U \in \T$ such that $x\in U \subset S$.
		\item The \textbf{interior} of $S$, is defined as $int(S)=\{ y\in S : y \text{ is an interior point of } S \}$
		\item Let $y\in X$ then $y$ is \textbf{adherent} to $S$ if for each open neighborhood $U$ of $y$, $U\cap S \neq \phi$.
		\item The \textbf{closure} of $S$ is defined as $\overbar{S}=\{ y\in X : y \text{ is an adherent point of }S \}$.
	\end{enumerate}
\end{definition}

\vspace{0.15in}

\begin{remark}
	Same as the case for metric spaces the following always hold for $S\subset X$ where $X$ is a topological space. \\
	\begin{equation}\nonumber
		int(S) \subset S \subset \overbar{S}
	\end{equation}\\
	This is held naturally by the definition we stated above.
\end{remark}

\vspace{0.15in}

\begin{theorem}
	For a topological space $(X,\T)$ where $S \subset X$ , 
	\begin{enumerate}[label=\arabic*)]
		\item $S$ is open $\iff$ $S=int(S)$
		\item $int(S)$ is open.
		\item $S$ is closed $\iff$ $S=\overbar{S}$.
		\item $\overbar{S}$ is closed.\\
	\end{enumerate}
\end{theorem}

\begin{proof}
We will just prove for the first two theorems as we can use similar arguments for closed sets.
\begin{enumerate}[label=\arabic*)]
	\item $S$ is open $\iff$ $S=int(S)$
	\begin{enumerate}[label=\roman*)]
		\item $S$ is open $\implies$ $S=int(S)$
		\vskip 0.5ex Let $x\in S$. In order to show the above implication we have to show that for any $x$, there exists an open set $U$ such that $x\in U \subset S$. But by assumption we can always choose such $U$ to be $S$ which is open and also a trivial subset of $S$. 
		\item $S$ is open $\impliedby$ $S=int(S)$
		\vskip 0.5ex As $S=int(S)$, for each $x\in S$ there exists an open neighborhood $U_x$, i.e $x\in U_x \subset S$. Then $S=\bigcup\limits_{x \in  S}U_x$ as 
		\begin{enumerate}
			\item $S=\bigcup\limits_{x \in  S}\{x\}\subset \bigcup\limits_{x \in  S}U_x$
			\item As for all $x\in S : U_x \subset S$ which implies that $\bigcup\limits_{x \in  S}U_x \subset S$
		\end{enumerate}
	As $S$ is an union of open neighborhoods, it is open.\\
	\end{enumerate}
	\item $int(S)$ is open.
	\vskip 0.5ex It suffices to show that $int(S) \subset int(int(S))$. Let $x\in int(S)$, then there exists an open set $U$ such that $x\in U \subset S$ by definition. Let $y\in U$. Then $y\in U \subset S$ and as $U$ is open, $y\in int(S)$. This implies that $U\subset int(S)$, which again implies that $x\in U \subset int(S)$ where $U$ is open. Thus $x\in int(S) \implies x\in int(int(S))$.
\end{enumerate}
\end{proof}

\begin{example}
	Let $X=\{ a,b,c \}$ with the topology $\T$ given as $\T=\{ \phi,X,\{a\},\{c\},\{a,c\} \}$.
	\begin{enumerate}
		\item For $S=\{a,b\}$ what is $int(S)$?
		\vskip 0.5ex $\implies$ $int(S)=\{ a \}$
		\item For the same $S$ what is $\overbar{S}$?
		\vskip 0.5ex $\implies$ $\overbar{S}=\{a,b\}$ which can be derived in many ways : \\
		\begin{enumerate}
			\item Directly from the definition of adherent points.
			\vskip 0.5ex j
			\item Find all closed sets $E$ such that $S\subset E$.
			\vskip 0.5ex Closed sets : $\{ \phi, X, \{b,c\} , \{a,b\},\{b\} \}$. Then such $E$ shows up to be $X,\{a,b\}$. Then the closure of $\overbar{S} = X \cap \{a,b\} = \{a,b\}$.
			\item Fortunately as $X\setminus S = \{c\} \in \T$ we know $S$ is closed, thus $\overbar{S}=S$
		\end{enumerate}
	\end{enumerate}
\end{example}

\vspace{0.15in}

\begin{definition}
	Let $(X,\T)$ be a topological space where $S\subset X$. Then the \textbf{boundary} of $S$ denoted as $\partial S$ is defined as $\partial S = \overbar{S} \cap \overbar{X\setminus S}$.
\end{definition}

\vspace{0.15in}

\begin{definition}
	\nextline
	\begin{enumerate}
		\item For a topological space $X$, a \textbf{sequence} $\{x_n\}$ in $X$ is a function $f:\mathbb{N} \to X$ or its image, $x_n=f(n)$.
		\item A sequence $\{x_n\}_{n=1}^\infty$ in $X$ \textbf{converges} to $x\in X$ if for each open neighborhood $U$ of $x$, there exists $N>0$ such that $x_n\in U$ for all $n \geq N$.
	\end{enumerate}
\end{definition}

\vspace{0.15in}

\begin{example}
	Let $X=(\mathbb{R},\{\phi,\mathbb{R}\})$ be a topological space. Let $x_n$ be a sequence in $X$ where for all $n\in \mathbb{N} : x_n=0$. Under such circumstances, any $y\in\mathbb{R}$ is the limit of $\{x_n\}$. Let $U$ be an open neighborhood of $y$. As the trivial topology is given, $U=\mathbb{R}$. Take $N=1$, then for all $n\geq N=1$, as $x_n\in U \subset \mathbb{R}$, $y$ is the limit of $\{x_n\}$.
\end{example}

\vspace{0.15in}
\subsection{Subspaces}
\vspace{0.1in}

\begin{lemma}
	Let $(X,\T_X)$ be a topological space where $S\subset X$. Let $\T_S = \{U \cap S : S \in\T_X \}$. Then such $\T_S$ suffices to be a topology for $S$.
\end{lemma}

\vspace{0.15in}

\begin{definition}
Using the same notations from the previous Lemma, 
\begin{enumerate}[label=\arabic*)]
	\item $\T_S$ is called the \textbf{relative topology} of $S$ inherited from $(X,\T_X)$.
	\item Let $V \subset S$. $V$ is \textbf{(relatively) open} in $S$ if $V \in \T_S$.
	\item Similarly, $V$ is \textbf{(relatively) closed} in $S$ if $S\setminus V \in \T_S$.
	\item  $(S,\T_S)$ is a \textbf{subspace} of $(X,\T_X)$.
\end{enumerate}
\end{definition}

\vspace{0.15in}

\begin{example}
Consider $X=\mathbb{R}$ and $S=[0,\infty)\subset X$, where the usual topology $\T_X$ is given for $X$. Then $\T_S = \{ U \cap [0,\infty) : U \in \T_X \}$ which is the relative topology of $S$. Of course $(S,\T_S)$ is a subspace of $(X,\T_X)$. Then we can ask if we equip $S$ with its discrete topology $\T_{disc}$ does it also become a subspace of $X$? The answer is no, of course. Such question reduces into check whether the inherited relative topology $\T_S$ and the discrete topology $\T_{disc}$ is same, which is not. Consider $\{0\}\in \T_{disc}$, as any subset is open when the discrete topology is given, which is not open when $\T_S$ is given.
\end{example}

\vspace{0.15in}

\begin{example}
Consider $X=\{ a,b,c \}$ where the topology is given as $\T_X = \{ \phi, X, \{a\}, \{a,b\} \}$. If we let $S=\{b,c\} \subset X$ then we can directly derive the relative topology $\T_S = \{ U \cap \{b,c\} : U\in \T_X \}= \{ \phi, S , \{b\} \}$.
\end{example}

\vspace{0.15in}

\begin{example}
	Let $X$ be a set equipped with the discrete topology $\T_X=\mathcal{P}(X)$. Then the relative topology $\T_S$ of any $S\subset X$ is $\T_S =\{ U \cap S : U\in \mathcal{P}(X) \} = \mathcal{P}(S)$. Thus this tells us that the relative topology inherited by a discrete topology is also a discrete topology.
\end{example}

\vspace{0.15in}
\subsection{Continuous Functions}
\vspace{0.1in}

\begin{definition}
	Let $(X,\T_X)$ and $(Y,\T_Y)$ both be topological spaces where $f:X\to Y$ be a function. 
	\begin{enumerate}[label=\arabic*)]
		\item $f$ is \textbf{continuous} if for every open set $U$ of $Y$, $f^{-1}(U)$ is open in $X$. In other words, for any $U\in \T_Y \implies f^{-1}(U)\in \T_X$.
		\item $f$ is \textbf{continuous at} $x\in X$ if for an open neighborhood  $U$ of $f(x)$ there exists an open neighborhood $V$ of $x$ such that $x\in V \subset f^{-1}(U)$
	\end{enumerate}
\end{definition}

\vspace{0.15in}

\begin{lemma}
	$f$ is continuous $\iff$ $f$ is continuous at $x$ for all $x\in X$.
\end{lemma}

\vspace{0.15in}

\begin{theorem}
Let $X,Y,Z$ all be topological spaces where the topology is given as $\T_X,\T_Y,\T_Z$ respectively. If $f:X\to T$ and $g: Y\to Z$ are continuous functions then $g\circ f : X \to Z$ is also a continuous function.\\
\end{theorem}

\begin{proof}
	Let $U\in \T_Z$, and since $g$ is continuous $g^{-1}(U) \in \T_Y$. Also since $f$ is continuous, $f^{-1} \left( g^{-1}(U) \right) \in \T_X$. As $f^{-1} \left( g^{-1}(U) \right) = (g\circ f)^{-1}(U)$, for any $U\in \T_Z \implies (g\circ f)^{-1}(U) \in \T_X$ thus $g\circ f$ is also continuous.
\end{proof}

\vspace{0.15in}

Continuous functions gurantees that the preimage of an open set is also open in the domain, but it is not always guranteed that the image of an open set it open. Such maps which map open sets to open sets are called as \textbf{open maps}.\\

\begin{example}
	Consider $f:\mathbb{R}\to \mathbb{R}$ which maps $x\mapsto x^2$ where each $\mathbb{R}$ is equipped with the usual topology. Then $f$ is a continuous function but not an open map as $f((-1,1))=[0,1)\not\in \T_\mathbb{R}$ while $(-1,1) \in \T_\mathbb{R}$.
\end{example}

\vspace{0.15in}

\begin{definition}
	Let $X,Y$ be topological spaces where $f:X\to Y$ is a function. Then $f$ is a homeomorphism if : 
	\begin{enumerate}[label=\arabic*)]
		\item $f$ is a bijection.
		\item $f$ and $f^{-1}$ are both continuous.
	\end{enumerate}
\end{definition}

\end{document}
