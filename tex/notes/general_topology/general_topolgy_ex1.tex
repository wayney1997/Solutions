\documentclass[paper=a4, fontsize=11pt]{scrartcl}

%\usepackage[T1]{fontenc} 
\usepackage[english]{babel} 
\usepackage{amsmath}
\usepackage{amsfonts}
\usepackage{amsthm}
\usepackage{amssymb}
\usepackage{changepage}
\usepackage{titlesec}
\usepackage{sectsty} 
\sectionfont{\centering \normalfont \scshape}
\subsectionfont{\normalfont}
\subsubsectionfont{\normalfont}

\usepackage{fancyhdr} 
\pagestyle{fancyplain} 
\fancyhead{} 
\fancyfoot[L]{} 
\fancyfoot[R]{} 
\fancyfoot[C]{\thepage} 
\renewcommand{\headrulewidth}{0pt} 
\renewcommand{\footrulewidth}{0pt} 
\setlength{\headheight}{13.6pt} 

\usepackage{enumitem}
\newcommand{\subscript}[2]{$#1 _ #2$}

\newcommand{\horrule}[1]{\rule{\linewidth}{#1}} 
\newcommand{\ball}[2]{$B({#1};{#2})$}

\title{	
	\normalfont \normalsize 
	\textsc{Konkuk University Dept. Of Physics} \\ [25pt] %Konkuk University Dept. of Physics
	\horrule{1pt} \\[0.4cm] 
	\huge General Topology
	\horrule{1pt} \\[0.4cm] 
}

\author{Youngwan Kim} 
\date{\normalsize\today} 

\newtheorem{theorem}{Thm}
\newtheorem{definition}{Def}
%\newtheorem{examples}{Examples}
\newtheorem{example}{Ex}
\newtheorem{lemma}{Lem}
\begin{document}
	
\maketitle	

\section{Metric Spaces} 
\vspace{2.5ex}
\begin{definition}
	A \textbf{metric} $d$ defined on a set $X$ is a mapping $d:X \times X \to \mathbb{R}$ which has the following properties.
	\begin{enumerate}[label=(\subscript{m}{{\arabic*}})]
		\item $\forall x,y \in X : d(x,y) \geq 0$ , $d(x,y)=0 \iff x=y$
		\item $\forall x,y \in X: d(x,y)=d(y,x)$
		\item $\forall x,y,z \in X : d(x,y)+d(y,z) \geq d(x,z)$\\
	\end{enumerate}
\end{definition}
If such map is well defined on a certain set $X$, we can now introduce the notion of metric spaces, which is merely a set equipped with a well defined metric.
\\
\begin{definition}
	A \textbf{metric space} $(X,d)$ is a set $X$ equipped with a metric $d:X\times X \to \mathbb{R}$.
\end{definition}
\begin{example} 
	These are some simple examples of metric spaces.
	\begin{enumerate}
		\item $(\mathbb{R}^n,d)$ where $d(x,y)=$
		\item $(X,d)$ where $d(x,y)=0$ if $x=y$ and $d(x,y)=1$ if $x\neq y$ \textbf{(discrete metric)}
	\end{enumerate}
\end{example}
\vspace{1ex}
Now it is natural to think of how the metric would be affected under the subsets, and the following theorem states that the restriction of such metric is still a metric on the subsets. 
\begin{theorem}
	Let $(X,d)$ be a metric space and let $Y\subset X$. Then $d\restriction_{Y\times Y} : Y \times Y \to \mathbb{R}$ is a metric on $Y$.
\end{theorem}
\begin{proof}
	left as an exercise
\end{proof}
\vspace{2.5ex}
\begin{definition}
	Let $(X,d)$ be a metric space and $x\in X$, $r \in \mathbb{R}$ where $r>0$. The \textbf{open ball} \ball{x}{r} is defined as $\{y\in X : d(x,y)<r\}$ where $x$ is called the \textbf{center} of \ball{x}{r}, and $r$ is called the \textbf{radius} of \ball{x}{r}.\\
\end{definition}

\begin{example}
	Some basic examples of open balls on different metric spaces.
	\begin{enumerate}
		\item On $(\mathbb{R},d)$ where $d$ is the usual metric, \ball{x}{r}$=(x-r,x+r)$.
		\item On $(\mathbb{R},d)$ where $d$ is the discrete metric, \ball{x}{r}$=\{x\}$  if $r\in(0,1]$ and  \ball{x}{r}$=\mathbb{R}$  if $r>1$\\
	\end{enumerate}
\end{example}

\begin{lemma}
	Let $(X,d)$ be a metric space, where $x\in X$ and $r>0$. Then,
	\begin{enumerate}
		\item $\bigcap\limits_{r>0}$
	\end{enumerate}
\end{lemma}
\end{document}