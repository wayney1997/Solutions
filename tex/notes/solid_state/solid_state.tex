\documentclass[paper=a4, fontsize=11pt]{scrartcl}

%\usepackage[T1]{fontenc} 
\usepackage[english]{babel} 
\usepackage{amsmath}
\usepackage{amsfonts}
\usepackage{amsthm}
\usepackage{amssymb}
\usepackage{changepage}
\usepackage{titlesec}
\usepackage{tikz}
\usepackage{subcaption}
\usetikzlibrary{positioning}
\usepackage{sectsty} 
\sectionfont{\centering \normalfont \scshape}
\subsectionfont{\normalfont}
\subsubsectionfont{\normalfont}

\usepackage{fancyhdr} 
\pagestyle{fancyplain} 
\fancyhead{} 
\fancyfoot[L]{} 
\fancyfoot[R]{} 
\fancyfoot[C]{\thepage} 
\renewcommand{\headrulewidth}{0pt} 
\renewcommand{\footrulewidth}{0pt} 
\setlength{\headheight}{13.6pt} 

\usepackage{enumitem}
\newcommand{\subscript}[2]{$#1 _ #2$}

\newcommand{\boltz}{k_B}
\newcommand{\pder}[2]{\frac{\partial #1}{\partial #2}}
\newcommand{\dt}{\frac{d}{dt}}
\newcommand{\dder}[2]{\frac{d#1}{d#2}}
\newcommand{\horrule}[1]{\rule{\linewidth}{#1}} 
\newcommand{\overbar}[1]{
	\mkern 1.5mu \overline{\mkern-1.5mu\raisebox{0pt}[\dimexpr\height+0.5mm\relax]{$#1$}\mkern-1.5mu}\mkern 1.5mu
}
\newcommand*\dif{\mathop{}\!\mathrm{d}}
\newcommand{\expval}[1]{\langle #1 \rangle}
\newcommand{\tr}{\text{Tr }}


\title{	
	\normalfont \normalsize 
	\textsc{Konkuk University Dept. Of Physics} \\ [25pt] %Konkuk University Dept. of Physics
	\horrule{1pt} \\[0.4cm] 
	\huge Solid State Physics I \\
	\vspace{0.1in}
	\Large 2019 Spring Semester
	\horrule{1pt} \\[0.4cm] 
}

\author{Youngwan Kim} 
\date{\normalsize\today} 

\newtheorem{theorem}{Thm}
\newtheorem{definition}{Def}
%\newtheorem{examples}{Examples}
\newtheorem{example}{Ex}
\newtheorem{lemma}{Lem}
\newtheorem*{remark}{Remark}
\newtheorem*{recall}{Recall}

\begin{document}
	
\maketitle	

\vspace{0.25in}
\section{Introduction}
\vspace{0.25in}

Why is solid state physics, condensed matter physics is so important and has such value to be studied? From the early days mankind studied alot about various solids and their properties, for instance how well do they stretch, or how well do they accept heat energy, or how well do they conduct electricity and so on. 

\vspace{0.25in}
\section{Thermal Properties : Heat Capacity of Solids}
\vspace{0.25in}

We mainly discuss about the specific heat or heat capacity of solids, where we will briefly define the \textbf{heat capacity} as \\

\begin{equation}\nonumber
	C = \pder{U}{T}
\end{equation}\\

without any statistical derivation, and also we won't give difference between $C_V$ and $C_P$. Before the concept of \textbf{phonons} existed, there were several attempts to understand the heat capacity of solids, and one of those was the \textbf{Dulong-Petit law}.

\vspace{0.15in}
\subsection{The Dulong-Petit Law : Classical Approach}
\vspace{0.15in}

The Dulong-Petit law proposed in 1819, was an experimental rule derived by room temperature data of solids. They empirically discovered that molar heat capacity of solids had values of \\

\begin{equation}\nonumber
	C = 3R
\end{equation}\\

where $R$ is the gas constant. Using the relation of the Boltzmann constant and the gas constant we can also derive that the heat capacity per atom should be \\

\begin{equation}\nonumber
	C = 3 \boltz
\end{equation}\\

How can one explain such relation? It is a somehow interesting result as it states that under room temperature heat capacity of solids are constant, independent with the temperature. \\

This can be explained by using a theorem from classical statistics, the \textbf{equipartition theorem} stated by Boltzmann. The equipartition theorem states that any quadratic term of $p,q$ in the Hamiltonian, for instance $T \sim p^2$ or $V \sim q^2$, will yield an expectation value of $\frac{1}{2} \boltz T$. This implies that the total energy of a state is dependent on the degree of freedom such state possesses. Let us denote the degree of freedom as $f$ then the equipartition theorem states that \\

\begin{equation}\nonumber
	E = \frac{f}{2} \boltz T \implies C = \pder{E}{T} = \frac{f}{2} \boltz 
\end{equation}\\

which shows up to be a constant. It somehow explains the Dulong-Petit law as we've shown that the heat capacity should be a constant independent of the temperature. However we yet explained why the coefficient is $3$, which is to explain that $f=6$. We can give a qualitative explanation to this, if we consider a simple model only regarding vibrational freedom of each 3 dimensions it somehow explains $f=6$. \\

We need to keep in mind that the Dulong-Petit law was an empirical result of \textit{room temperature data}, which means that for other temperature regions it is not guranteed that such law should hold. As soon as mankind became accessible to lower temperatures, thanks to the discovery and development of methods of liquifying air, at lower temperature regions the Dulong-Petit law didn't apply so well.\\

\vspace{0.15in}
\subsection{Einstein Model : Harmonic Oscillators}
\vspace{0.15in}

Einstein made a further step towards this problem by making some asusmptions for solids : \\

\begin{itemize}
	\item Each atoms are considered as independent 3D quantum harmonic oscillators. 
	\item All atoms oscillate with the same frequency, $\omega_E$.\\
\end{itemize}

which we can imagine of solids as a lattice of springs with all of them having identical spring constants. As we know the energy spectrum of single simple harmonic oscillators, we can derive the expectation value of the total energy for $3N$ of the oscillators per mole of atom as : \\

\begin{equation}\nonumber
	\expval{E} = 3N \left( \expval{n} + \frac{1}{2}  \right) \hbar \omega_E
\end{equation}\\

where $\expval{n}$ denotes the mean quantum number, which will follow the Einstein-Bose statistics as lattice vibration has bosonic nature (?). Thus the mean quantum number $\expval{n}$ can be expressed as : \\

\begin{equation}\nonumber
	\expval{n} = \frac{1}{\exp(\frac{\hbar \omega_E}{\boltz T})-1}
\end{equation}\\

thus we eventually derive the expectation values of the total energy as : \\

\begin{equation}\nonumber
		\expval{E} =   \frac{3N  \hbar \omega_E }{\exp(\frac{\hbar \omega_E}{\boltz T})-1}   + \frac{3N}{2}  \hbar \omega_E
\end{equation} \\

Thus now we can derive the heat capacity for the Einstein model :\\

\begin{equation}\nonumber
\begin{split}
	C &= \pder{\expval{E}}{T} = 3N  \hbar \omega_E  \cdot \pder{}{T}\left(  \frac{ 1}{\exp(\frac{\hbar \omega_E}{\boltz T})-1} \right) \\[2.5ex]
	&= 3R \left( \frac{\hbar \omega_E}{\boltz T} \right)^2 \cdot \frac{\exp(\hbar \omega_E /\boltz T)}{(\exp(\hbar \omega_E /\boltz T)-1)^2} \\
\end{split}
\end{equation}\\

First we can see that for high temperature regions, the Dulong-Petit law holds.\\

\begin{equation}\nonumber
\begin{split}
 	C &= 3R \left( \frac{\hbar \omega_E}{\boltz T} \right)^2 \cdot \frac{\exp(\hbar \omega_E /\boltz T)}{(\exp(\hbar \omega_E /\boltz T)-1)^2} \\[2.5ex]
 	&\sim 3R \left( \frac{\hbar \omega_E}{\boltz T} \right)^2 \cdot \frac{1 + (\hbar \omega_E /\boltz T)}{((1 + (\hbar \omega_E /\boltz T))-1)^2}  \\[2.5ex]
 	&= 3R \left( \frac{\hbar \omega_E}{\boltz T} \right)^2 \cdot \frac{1 + (\hbar \omega_E /\boltz T)}{(\hbar \omega_E /\boltz T)^2} \\[2.5ex]
 	&= 3R \left( 1+  \frac{\hbar \omega_E}{\boltz T}\right) \\[2.5ex]
 	&\sim 3R
\end{split}
\end{equation}\\

The high temperature limit means that, the temperature should be higher than a certain limit which is known as the Einstein temperature $\Theta_E$ : \\

\begin{equation}\nonumber
	\Theta_E = \frac{\hbar \omega_E}{\boltz}
\end{equation} \\

What happens at the low limit, i.e, $T\sim 0$ region? As the heat capacity derived by such model has transient terms of $T$, the heat capacity should vanish very fast near $T\sim 0$. But this again had some discrepancy between well known experimental results, which had a relation of $C \sim T^3$ near the low temperature limit. 

\vspace{0.15in}
\subsection{Debye Model : Phonons}
\vspace{0.15in}

Now Debye tweaks some assumptions of the Einstein model in order to deal with the $T\sim 0$ region anomaly of the preceeding model. He fixed the condition that all atoms should oscillate with the same frequency, but rather that they will follow a certain dispersion relation. This was based on Debye's idea that atoms can't vibrate independently, but rather collectively. This lead to the emergence of the concept of \textbf{phonons}, which can be thought as the quanta of energy that such vibrational lattices can accept. The following would sum up what is different between Einstein's model and Debye's model. \\

\begin{itemize}
	\item Einstein : (phonon energy) $\times$ $\expval{n}$
	\item Debye : (phonon energy) $\times$ $\expval{n}$ $\times$ (number of modes)\\
\end{itemize}

The dispersion relation of Debye model is given as $\omega(k) = v_s k$ where $v_s$ is the sonic speed. The total energy of the phonons at temperature $T$ can be written as : \\

\begin{equation}\nonumber
\begin{split}
	U &= \sum_{k,p} \expval{n_{k,p}} \hbar \omega_{k,p} \\[2.5ex]
	&= \sum_{k,p} \frac{\hbar \omega_{k,p} }{\exp(\hbar \omega_{k,p}  / \tau) - 1}  \\[2.5ex]
	&= \sum_{p} \int \dif \omega \mathcal{D}_p(\omega) \left( \frac{\hbar \omega_{k,p} }{\exp(\hbar \omega_{k,p}  / \tau) - 1}  \right)
\end{split}
\end{equation}\\

where a new function $\mathcal{D}_p(\omega)$ of $\omega$ is introduced, is known as the \textbf{density of states (DoS)}. As we've derived an expression for the total energy we can derive the heat capacity as : \\

\begin{equation}\nonumber
\begin{split}
	C = \pder{U}{T} &= \pder{}{T} \left( \sum_{p} \int \dif \omega \mathcal{D}_p(\omega) \left( \frac{\hbar \omega_{k,p} }{\exp(\hbar \omega_{k,p}  / \tau) - 1}  \right) \right) \\[2.5ex]
	&= \boltz  \sum_{p} \int \dif \omega \mathcal{D}_p(\omega) \frac{x^2 \exp(x)}{(\exp(x)-1)^2}
\end{split} 
\end{equation}\\

which shows us that the problem of obtaining heat capacity from this model boils down to deriving density of states. We will derive the density of states for $d=1,2,3$ dimensions.

\subsubsection{DoS : 1D System}

\subsubsection{DoS : 2D System}

\subsubsection{DoS : 3D System}

\vspace{0.25in}
\section{Electrons in Metal}
\vspace{0.25in}

We mainly focused on how vibrational quanta, or phonons affect the properties of a solid.  But we all know that we can't just ignore the bunch of electrons that are inside in it. In this section we mainly regard the effects of electrons in solids, and introduce two different approaches that were attempted, and how they predicted some important physical behaviors.

\vspace{0.15in}
\subsection{Drude Theory : Kinetic Approach}
\vspace{0.15in}

Drude applied the kinetic theory  of gases directly into the free electron gas in solids in order to understand how electrons affect solid properties. He made some assumptions, mainly derived from the kinetic theory of gases. \\

\begin{itemize}
	\item  Electrons will have a scattering time $\tau$. 
	\vskip 0.5ex
	$\implies$ The probability of such scattering occuring within a $dt$ time interval is $dt/\tau$.
	\item After a scattering occurs, the momentum of the scattered electron will be $\mathbf{p}=0$
	\item The electrons will respind to external electromagnetic fields, $\mathbf{E}$ and $\mathbf{B}$.\\
\end{itemize}

The first two assumptions were the direct application of kinetic theory of gas to free electrons, while the last assumption is somehow trivial; electrons react to electromagnetic fields. From the assumption we can derive the equation of motion of electrons : \\

\begin{equation}\nonumber
	\mathbf{p}(t+dt) = \left( 1-\frac{dt}{\tau} \right) (\mathbf{p}(t)+\mathbf{F}dt) + \frac{dt}{\tau} \cdot 0
\end{equation}\\

the first term states the situation where it doesn't scatter and the next one is when it scatters. Then we can simplify the above equation into : \\

\begin{equation}\nonumber
\frac{\mathbf{p}(t+dt) - \mathbf{p}(t)}{dt} = \mathbf{F}(t) - \frac{\mathbf{p}(t)}{\tau}
\end{equation}\\

which can be then simplified into : \\

\begin{equation}\nonumber
	 \frac{d\mathbf{p}}{dt} = \mathbf{F}(t) - \frac{\mathbf{p}(t)}{\tau}
\end{equation}\\

As the $\mathbf{F}$ term for electrons would be the Lorentz force by the third assumption of Drude, we then express the above equation in terms of fields : \\

\begin{equation}\nonumber
	\frac{d\mathbf{p}}{dt} = -e \left( \mathbf{E}  + \mathbf{v}(t) \times \mathbf{B}\right) - \frac{\mathbf{p}(t)}{\tau}
\end{equation}\\

We will mainly use this equation to derive physical results from this model. Let us consider two cases : where $\mathbf{B}=0$ and  $\mathbf{B}\neq0$.

\vspace{0.15in}
\subsubsection{Drude Theory : Electric Fields}
\vspace{0.15in}

We consider situations where $\mathbf{B}=0$, thus only considering electric fields. Then the equation of motion we derived can be expressed as : \\

\begin{equation}\nonumber
	\frac{d\mathbf{p}}{dt} = -e  \mathbf{E}   - \frac{\mathbf{p}(t)}{\tau}
\end{equation}\\

and we will consider steady states where $d\mathbf{p}/dt = 0$. Then the above equation yields,\\

\begin{equation}\nonumber
\begin{split}
 -e  \mathbf{E}  &= \frac{\mathbf{p}(t)}{\tau} \\[2.5ex]
  \mathbf{v} &= - \frac{e \tau}{m} \mathbf{E} \\
\end{split}
\end{equation}\\

We can define and derive the \textbf{Drude conductivity} $\sigma$ and \textbf{Drude mobility} $\mu$, which are the coefficients that satisfy : \\

\begin{equation}\nonumber
	  \mathbf{j} = \sigma \mathbf{E} \qquad \mathbf{v} = \mu \mathbf{E}
\end{equation}\\

and from the previous derivation we can show that both are \\

\begin{equation}\nonumber
\sigma = \frac{e^2 n \tau }{m}\qquad \mu = -\frac{e\tau}{m}
\end{equation}\\

where we used the relation of $\mathbf{j} = -en \mathbf{v}$. This implies that if we know the electron mass and charge, when we measure the conductivity of a metal we can derive the density of electrons in the metal. \\

Also this result can be applied to understand resistivity of metals. Resistivity of metals mainly occur from two origins : phonons and impurities. \textbf{Matthiessen's rule} uses the scattering probability differed by those two origins. The probability that an electron wil be scattered by phonons in $dt$ interval can be expressed as $dt/\tau_{L}$ where the $L$ denotes the L of lattice, and by impurities, $dt/\tau_{I}$. Then the total probability would be expressed as \\

\begin{equation}\nonumber
	\frac{dt}{\tau} = \frac{dt}{\tau_L} +  \frac{dt}{\tau_I} \implies 	\frac{1}{\tau} = \frac{1}{\tau_L} +  \frac{1}{\tau_I}
\end{equation}\\

and using the Drude mobility we obtain \\

\begin{equation}\nonumber
	\frac{1}{\tau} = \frac{1}{\tau_L} +  \frac{1}{\tau_I} \implies  \frac{1}{\mu} = \frac{1}{\mu_L} +  \frac{1}{\mu_I} 
\end{equation}\\

and in terms of resistivity, \\

\begin{equation}\nonumber
	\rho = \rho_L + \rho_I
\end{equation}\\

which well explains that the total resistivity can be expressed by the sum of resistivity occured by two different sources. An interesting fact is that $\rho_L(T)$ is dependent of temperature but $\rho_I$ is a constant dependent with the temperature. Also when $T\to 0$, $\rho_L(T) \to 0$ which implies that the residual resistivity is $\rho(0)= \rho_I$. This implies that $\rho_L(T) = \rho(T) - \rho_I $ is same for different specimens of metal, while $\rho_I$ should vary a lot by the purity of the specimen. Based on this, we use the \textbf{resistivity ratio} as an indicator of sample purity, which is defined as the ratio of residual resistivity to room temperature resistivity. Note that the purer the smaple is the higher this value becomes. For instance, exceptionally pure specimen can have resistivity ratio of an order of $\sim 10^6$ while common alloy has a value around $1.1$. \\

We can also think of the mean distance an electron can travel through without scattering, or the \textbf{mean free path} as the scattering time multiplied with the Fermi velocity, \\

\begin{equation}\nonumber
	l = v_F \tau 
\end{equation}\\
 
The purer the metal is, the longer $l$ becomes. The reason we used the Fermi velocity to calculate such physical quantity is because only the electrons in the Fermi level would get excited.\\
 
\vspace{0.15in}
\subsubsection{Drude Theory : Electromagnetic Fields}
\vspace{0.15in}

Now we consider the effects of $\mathbf{B}$, and again consider steady states. Then, \\

\begin{equation}\nonumber
\begin{split}
		-\frac{m\mathbf{v}(t)}{e \tau} &=  \mathbf{E}  + \mathbf{v}(t) \times \mathbf{B} \\[2.5ex]
		 \mathbf{E} &= - \left(  \mathbf{v}(t) \times \mathbf{B} + \frac{m\mathbf{v}(t)}{e \tau} \right)\\[2.5ex]
		  \mathbf{E} &= \left(  \frac{1}{ne}\mathbf{j} \times    \mathbf{B} +\frac{m}{ne^2 \tau}  \mathbf{j}\right)
\end{split}
\end{equation}\\

where we again used the relations of $\mathbf{j} = -en \mathbf{v}$. Let us think of a simple case where $\mathbf{B}=(0,0,B)$, which means that there is only the z-axis component for the magnetic field. Then, \\

\begin{equation}\nonumber
\frac{1}{ne} \mathbf{j} \times \mathbf{B} = \frac{1}{ne} \left( j_y B , -j_x B , 0 \right)
\end{equation}\\

Thus, \\

\begin{equation}\nonumber
\mathbf{E} = \frac{1}{ne} \left( j_y B , -j_x B , 0 \right) + \frac{1}{\sigma_0} (j_x,j_y,j_z) 
\end{equation}\\

where we defined $\sigma_0$ as $ ne^2\tau/m $. This can be again simplified as $\mathbf{E} = \tilde{\rho} \mathbf{j}$, where we define the tensor $\tilde{\rho}$ as the \textbf{(magneto) resistivity tensor}, \\

\begin{equation}\nonumber
	\tilde{\rho} = \frac{1}{\sigma_0}
	\begin{pmatrix}
	1 & \tau/\omega_c & 0 \\
	-\tau/\omega_c & 1 & 0\\
	0 & 0 & 1 \\
	\end{pmatrix}
\end{equation}\\

where $\omega_c$ is the \textbf{cyclotron frequency}, defined as $eB/m$. Now we define the \textbf{Hall coefficient}, $R_H$ as \\

\begin{equation}\nonumber
	R_H = \frac{\rho_{yx}}{B} = -\frac{B}{ne}
\end{equation}\\

where the last equation is using the results of the Drude theory. This implies that we can measure the density of electrons in a metal via Hall effect. Or conversely, if you know the electron density of the sample we can calculate the magnetic field strength. We call an apparatus which utilizes such method, a \textbf{Hall sensor}.

\vspace{0.15in}
\subsubsection{The Success : The Wiedermann-Franz Law}
\vspace{0.15in}

Besides the electromagnetic effects the Drude model predicted, it also explained the empirical \textbf{Wiedermann-Franz Law}, which states that\\

\begin{equation}\nonumber
	\frac{K}{\sigma T} = L
\end{equation} \\

where $K$ is the thermal conductivity, $\sigma$ is the electrical conductivity and $L$ is a constant named the Lorenz number. Based on the Drude theory, we can derive $K$ as \\

\begin{equation}\nonumber
	K = \frac{1}{3} C v l = \frac{1}{3} n c v^2 \tau
\end{equation}\\

and as Drude theory was based on the kinetic theory of gas, we use the $v$ from the results of the Maxwell speed distribution. And the heat capacity per atom $c$ also follows the classical limit :\\

\begin{equation}\nonumber
	v = \sqrt{\frac{8RT}{\pi M}} \qquad c = \pder{}{T} \left( \frac{3}{2} \boltz T \right) = \frac{3}{2} \boltz	 
\end{equation}\\

Then we can derive that \\

\begin{equation}\nonumber
	K = \frac{1}{3} n \left(\frac{3}{2} \boltz\right) \left( \frac{8RT}{\pi M} \right) \tau = 4 \boltz \frac{n R T }{\pi M} \tau
\end{equation}\\

which leads to the Lorenz number : \\

\begin{equation}\nonumber
\begin{split}
	L &= \frac{K}{\sigma T} \\[2.5ex]
	& =   \frac{1}{\sigma T} \cdot 4 \boltz \frac{ nR T }{\pi M} \tau \\[2.5ex]
	&= \frac{m}{e^2 n \tau T}  \cdot 4 \boltz \frac{ nR T }{\pi M} \tau \\[2.5ex]
	&= \frac{4 \boltz^2}{\pi e^2 } \sim 1.11 \times 10^{-8} \text{  WhattOhm/K$^2$}
\end{split}
\end{equation}\\

where we used the Drude conductivity to derive the final result. This was a huge success at that time as it explained the Wiedermann-Franz law in a quite precise order. Then one might think, how did this rough model well predict such thermal properties of metals? We all know that electrons do not follow the kinetic theory of gas, then how did it explain such phenomena so well?\\

We will deal with the details later but the direct answer with that the heat capacity was overrated and the velocity was underrated, which kind of balenced out, leading the right results.\\

\vspace{0.15in}
\subsubsection{The Failure : The Peltier Effect and Seeback Coefficient }
\vspace{0.15in}

However, despite of the unintended success of Drude model giving a right value of the Lorenz number, it failed to do so for the \textbf{Seeback coefficient}. By Drude's theory, the thermal current is \\

\begin{equation}\nonumber
	j^q = 
\end{equation} \\

In this case Drude model failed to explain such phenomena as 

\vspace{0.15in}
\subsection{Sommerfeld Theory : Statistical Approach}
\vspace{0.15in}

Sommerfeld had a quite different perspective dealing with electrons in metal with Drude, where Drude applied the classical kinetic theory to electrons, he take a step further by using quantum statistics, based on his idea that Drude’s theory of metals could easily be generalized to incorporate Fermi statistics.\\

Let us first introduce some basic Fermi statistics. A system which follows certain statistics should yield a 

\vspace{0.25in}
\section{Crystal Structures}
\vspace{0.25in}

\vspace{0.25in}
\section{Crystal Binding}
\vspace{0.25in}

\vspace{0.15in}
\subsection{Inert Gas Crystals : Cohesive Energy}
\vspace{0.15in}

Inert gases are generally neutral and non polarized. But just saying so doesn't really helps us explaining how inert gas bonds and form crystals. At some moment a dipole can be induced by instantaneous charge splitting, and such induced dipole would induce another dipole and so on, which would eventually make every inert gas molecules interact as dipoles. A model depicting such situation is the Van der Waals-London interaction model. \\

The Van der Waals-London equation states that the attraction between induced inert gas dipoles follow a potential of an order of $\sim R^{-6}$. We also need a repulsive term, as if there was only the attraction term, we can't fully explain how the atoms stably bond without collapsing into each other. Also repulsion must occur due to the Pauli exclusion principle, as when two atoms get closer the electrons get closer too, which implies a higher chance of same states getting superpositioned. Conversely with the case of Van der Waals-London interaction, the repulsive term is derived somewhat phenomenologically. 

\vspace{0.15in}
\subsection{Ionic Crystals}
\vspace{0.15in}

\vspace{0.15in}
\subsection{Covalent Crystals}
\vspace{0.15in}

\vspace{0.15in}
\subsection{Metals}
\vspace{0.15in}

\end{document}